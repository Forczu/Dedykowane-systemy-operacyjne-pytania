% !TeX spellcheck = pl_PL
\documentclass[a4paper,twoside]{article}
\usepackage{polski}
\usepackage[utf8]{inputenc}
\usepackage{graphicx}
\usepackage{amsmath}

\usepackage[unicode, bookmarks=true]{hyperref} %do zakładek
\usepackage{tabto} % do tabulacji
\NumTabs{6} % globalne ustawienie wielkosci tabulacji
\usepackage{array}
\usepackage{multirow}
\usepackage{array}
\usepackage{dcolumn}
\usepackage{bigstrut}
\usepackage{color}
\usepackage[usenames,dvipsnames]{xcolor}
\usepackage{svg}
\usepackage{xfrac}
\usepackage{floatrow}
% Table float box with bottom caption, box width adjusted to content
\newfloatcommand{capbtabbox}{table}[][\FBwidth]
\usepackage{blindtext}
\usepackage{enumerate}
\usepackage{wrapfig}


\setlength{\textheight}{24cm}
\setlength{\textwidth}{15.92cm}
\setlength{\footskip}{10mm}
\setlength{\oddsidemargin}{0mm}
\setlength{\evensidemargin}{0mm}
\setlength{\topmargin}{0mm}
\setlength{\headsep}{5mm}

\newcolumntype{M}[1]{>{\centering\arraybackslash}m{#1}}
\newcolumntype{N}{@{}m{0pt}@{}}

% === Reset inkrementacji sekcji przy nowym parcie === %
\usepackage{titlesec}

\makeatletter
\@addtoreset{section}{part}
\makeatother
\titleformat{\part}[display]
{\normalfont\LARGE\bfseries\centering}{}{0pt}{}

% === DEFINICJA ZIELONEGO ==================== %
\definecolor{Gurin}{rgb}{0, 0.35, 0}

% === MAKRODEFINICJA POPRAWNEJ I ZŁEJ ODPOWIEDZI ==================== %
\newcommand{\Tak}[1] {
	\color{Gurin}{#1}
}
\newcommand{\Nie}[1] {
	\color{Red}{#1}
}

% === MAKRODEFINICJA PYTANIA I ODPOWIEDZI =========================== %
% ** Pierwszy argument to treść pytania
% ** Kolejne argumenty to:
% * parzyste - Tak lub Nie
% * nieparzyste - Treści odpowiedzi
\newcommand{\question}[9] {
	\textbf{#1}
	\begin{enumerate}[a.]
		\ifnum\pdfstrcmp{#2}{Tak}=0
			\Tak{\item #3}
		\else
			\Nie{\item #3}
		\fi
		\color{black}
		\ifnum\pdfstrcmp{#4}{Tak}=0
			\Tak{\item #5}
		\else
			\Nie{\item #5}
		\fi
		\color{black}
		\ifnum\pdfstrcmp{#6}{Tak}=0
			\Tak{\item #7}
		\else
			\Nie{\item #7}
		\fi
		\color{black}
		\ifnum\pdfstrcmp{#8}{Tak}=0
			\Tak{\item #9}
		\else
			\Nie{\item #9}
		\fi
	\end{enumerate}
}
\begin{document}
\bibliographystyle{plain}

% ************************************************************
% *** WZÓR PYTANIA DO SKOPIOWANIA

%\item \question{}%
%{Tak}{}%
%{Nie}{}%
%{}{}%
%{}{}

% ************************************************************

\begin{titlepage}
\title{\huge Dedykowane Systemy Operacyjne - zbiór pytań}
\author{\large zebrali SonMati i Ervelan}
\maketitle
\end{titlepage}

% =============== PYTANIA ==================================== %
% ======== WINDOWS =========================================== %
\part{Windows}
	
	% --- AD - Active Directory ------------ %
	\input{./Tex/Win_1_AD.tex}
	
	% --- GPO - Obieky Zasad Grup ---------- %
	\newpage
\section{Obiekty Zasad Grup (GPO)}
	\begin{enumerate}
		\item \question{Na jakich poziomach w Active Directory mogą być przypisywane obiekty GPO?}%
		{Tak}{Lokalnie}% 
		{Nie}{Na poziomie lokacji}%
		{Tak}{Na poziomie domeny}%
		{Tak}{Na poziomie jednostki organizacyjnej}%
		
		\item \question{Aby wyświetlić wynikowy zestaw zasad dla użytkownia Sysop należy użyć polecenia:}%
		{Nie}{gpresult /gpo Sysop}%
		{Nie}{gpresult /?}%
		{Tak}{gpresult /user Sysop}%
		{Nie}{gpresult /u Sysop}
		
		\item \question{Wskaż prawdziwe zdania dotyczące GPO}%
		{Nie}{Akronin GPO rozwija się jako Group Policy Operation}%
		{Tak}{Za pomocą GPO Standard Desktop można zabronić dostępu do Panelu Sterowania}%
		{Nie}{Dane jednego GPO mogą być przypisane tylko jednej jednostce organizacyjnej}%
		{Nie}{Nie da się wyłączyć stosowania zasad GPO danej jednostki organizacyjnej bez usuwania GPO lu łącza obiektu}
		
		\item \question{Gdzie w rejestrze systemowym można znaleźć wpisy wynikające z GPO?}%
		{Tak}{HKEY LOCAL MACHINE (HKLM)}%
		{Nie}{HKEY CLASSES ROOT (HKCR)}%
		{Tak}{HKEY CURRENT USER (HKCU)}%
		{Nie}{HKEY USERS (HKU)}
		
		\item \question{W jaki sposób można modyfikować domyślne przetwarzanie obiektów zasad grupy?}%
		{Tak}{Blokując dziedziczenie zasad grupy}%
		{Nie}{Definiując warunkowe wprowadzanie ustawień.}%
		{Tak}{Wyłączając przetwarzanie konkretnego łącza GPO}%
		{Tak}{Wyłączając nadpisywanie ustawień wprowadzanych przez konkretne łącze GPO.}
		
		\item \question{Group Policy Management Console umożliwia:}%
		{Nie}{Wszystkie funkcje konsoli Power Shell, oraz dodatkowo funkcje zarządzania obiektami GPO}%
		{Tak}{Stworzenie kopii zapasowej obiektów GPO}%
		{Tak}{Łatwiejsze zarządzanie obiektami GPO, dzięki graficznemu interfejsowi użytkownika}%
		{Nie}{Tworzenie logów każdej operacji użytkownika w wybranej przez administratora grupie}
		
		\item \question{System Windows w ramach zarządzania GPO umożliwia:}%
		{Tak}{Filtrowanie ustawień GPO - wyłączenie stosowania określonych zasad GPO}%
		{Tak}{Wymuszanie stosowania zasad GPO}%
		{Tak}{Przeglądanie wdrażania elementów GPO dla danej jednostki organizacyjnej}%
		{Tak}{Blokowanie dziedziczenia ustawień obiektów GPO}
		
		\item \question{Które narzędzia służą do tworzenia i zarządzania GPO?}%
		{Tak}{Konsola Group Policy Management}%
		{Nie}{narzędzie gpadd}%
		{Tak}{Group Policy Object Editor z Active Directory Users and Computers}%
		{Nie}{narzędzie gpomod}
		
		
	\end{enumerate}
		
	% --- Instalacja Zdalna ---------------- %
	%\item \question{}%
%{Tak}{}%
%{Nie}{}%
%{}{}%
%{}{}

% !TeX spellcheck = pl_PL
\newpage
\section{Windows Instalacja zdalna}
	\begin{enumerate}
		\item \question{Windows Deployment Services (WDS):}%
		{Nie}{Pozwala na przygotowanie obrazów dysków do zautomatyzowania lokalnej instalacji systemu Windows.}%
		{Tak}{Pozwala na instalację systemu Windows przez sieć.}%
		{Nie}{Możliwe jest instalowanie przez sieć wyłącznie systemów serwerowych np. Windows Server 2008.}%
		{Nie}{Możliwa jest zdalna instalacja (przez sieć) systemu Linux wykorzystując system Windows Server.}
		\item \question{Windows Deployment Services wykorzystuje obrazy z rozszerzeniem:}%
		{Nie}{BIN}%
		{Nie}{MDF}%
		{Tak}{WIM}%
		{Nie}{ISO}
		\item \question{Format obrazów instalacyjnych wykorzystywany przez Windows Deployment Services to:}%
		{Nie}{VHD}%
		{Nie}{ISO}%
		{Nie}{IMG}%
		{Tak}{WIM}
		\item \question{Windows Deployment Services to:}%
		{Nie}{Tworzenie instalatorów dla programów na platformę .NET}%
		{Nie}{Instalację systemu Windows poprzez nośnik USB.}%
		{Tak}{Usługa pozwalająca na instalację systemu Windows przez sieć.}%
		{Nie}{Instalację i konfigurację aplikacji internetowej na serwerze IS.}
		\item \question{Windows Deployment Services (WDS) to technologia serwerowa, która pozwala na:}%
		{Nie}{Zdalne logowanie do systemu.}%
		{Tak}{Sieciową instalację systemu operacyjnego.}%
		{Tak}{Instalację systemu operacyjnego bez płyty instalacyjnej typu CD lub DVD.}%
		{Nie}{Lokalne monitorowanie systemu operacyjnego chroniąc przed złośliwym oprogramowaniem.}
		\item \question{Aby możliwa była zdalna istalacja, to maszyna kliencka może uruchamiać się z:}%
		{Nie}{dysku twardego}%
		{Tak}{karty sieciowej}%
		{Nie}{napędu CD / DVD}%
		{Nie}{nie ma to znaczenia}
		\item \question{Jakie elementy są wymagane do poprawnej pracy WDS?}%
		{Nie}{Windows Server w wersji 2008 lub wyższej.}%
		{Tak}{Usługa Windows Deployment Services zainstalowana na serwerze udostępniającym obrazy do instalacji.}%
		{Nie}{Sprzęt sieciowy obsługujący protokół WDS (ro\textsl{uter, switch, karta siecio}wa)}%
		{Tak}{Kontroler domeny, serwer DNS, serwer DHCP}
		\item \question{Które z poniższych zdań na temat wymagań instalacji zdalnej jest prawdziwe?}%
		{Tak}{Serwer WDS musi być członkiem domeny Active Directory.}%
		{Tak}{W sieci musi znajdować się serwer DNS.}%
		{Tak}{W sieci musi znajdować się serwer DHCP.}%
		{Nie}{Serwery DHCP i DNS muszą być niezależne od serwera WDS.}
		\item \question{Wykorzystując zdalną instalację systemu Windows:}%
		{Tak}{Jeden serwer umożliwia instalację wielu wersji systemu (użytkownik może sam wybrać).}%
		{Nie}{Jeden serwer pozwala na instalację tylko jednej wersji systemu (np. Ultimate)}%
		{Tak}{Pliki z obrazem systemu muszą być dostępne na serwerze.}%
		{Nie}{Do komputera na którym instalowany jest system trzeba włożyć płytę z obrazem systemu (ale konfiguracja instalowanego systemu jest (...))}
		\item \question{Jakie warunki muszą być spełnione by można było pomyślnie zainstalować usługę WDS?}%
		{Nie}{Sieć musi być połączona z Internetem.}%
		{Tak}{Komputer musi być członkiem domeny Active Directory.}%
		{Tak}{W sieci musi znajdować się serwer DNS.}%
		{Tak}{W sieci musi znajdować się serwer DHCP.}
		\item \question{Aby możliwe było wykorzystanie Windows Deployment Services konieczny jest:}%
		{Tak}{Serwer DHCP wskazujący lokalizację pliku uruchomieniowego.}%
		{Nie}{Serwer FTP z którego będą pobierane pliki instalacyjne.}%
		{Nie}{Obraz instalacyjny z systemem Windows 7 w edycji co najmniej Professional.}%
		{Tak}{Obraz środowiska Windows PE.}
		\item \question{Mechanizm WDS umożliwia:}%
		{Nie}{Zdalną instalację systemów z obrazów płyt .iso}%
		{Tak}{Zdalną instalację systemów Windows.}%
		{Nie}{Zdalne zarządzanie zainstalowanymi systemami Windows.}%
		{Tak}{Zdalną instalację systemów z obrazów płyt .wim}
		\item \question{Wskaż poprawne zdania dotyczące WDS:}%
		{Tak}{Proces instalacji systemu na komputerze klienckim rozpoczyna się od przesłania po sieci obrazu bardzo uproszczonego systemu operacyjnego (...) głównego instalatora.}%
		{Nie}{Serwer w momencie instalowania usługi WDS automatycznie instaluje obrazy płyt używane do instalacji systemu po sieci.}%
		{Tak}{Aby zainstalować na komputerze klienckim system Windows, używając mechanizmu WDS, należy ustawić w BIOSie bootowanie rozpoczynające (...) sieciowej.}%
		{Nie}{Używając WDS możemy instalować po sieci każdy system z rodziny Microsoft Windows i Linux.}
		\item \question{Wskaż poprawne zdania dotyczące WDS:}%
		{Nie}{Serwer w momencie instalowania usługi WDS automatycznie instaluje obrazy płyt używane to instalacji systemu po sieci}%
		{Tak}{Proces instalacji systemu na komputerze klienckim rozpoczyna się od przesłania po sieci obrazu bardzo uproszczonego systemu operacyjnego służącego do uruchomienia głównego instalatora}%
		{Nie}{Używając WDS możemy instalować po sieci każdy system z rodziny Microsoft Windows i Linux}%
		{Tak}{Aby zainstalować na komputerze klienckim system windows używając mechanizmu WDS należy ustawić w biosie boot'owanie rozpoczynające się od karty sieciowej}
				
		%\item \question{}%
		%{Tak}{}%
		%{Nie}{}%
		%{}{}%
		%{}{}
		
	\end{enumerate}
	
	% --- RAID ----------------------------- %
	%\item \question{}%
%{Tak}{}%
%{Nie}{}%
%{}{}%
%{}{}

% !TeX spellcheck = pl_PL
% ***************************************************************************
% --- Źródło było dość dziwne, sprawdzić potem jeszcze raz wszystkie pytania
% *************************************************************************** 
\newpage
\section{Windows RAID}
	\begin{enumerate}
		\item \question{Na komputerze posiadającym 5 dysków ma zostać zainstalowany system operacyjny Windows 2008 Server, który powinien zapewnić pracę z minimalnym prawdopodobieństwem utraty danych oraz łatwą administracją dyskami. Jaką konfigurację powinien wybrać administrator zakładając, że nie może użyć macierzy sprzętowych?}%
		{Nie}{wszystkie dyski spięte w mirror}%
		{Tak}{2 dyski spięte w mirror, pozostałe 3 dyski spięte w RAID5}%
		{Nie}{wszystkie 5 dysków spiętych w RAID5}%
		{Nie}{dyski spięte w spanned volume, 2 dyski spięte w mirror}
		\item \question{Maksymalna ilość dysków, które mogą ulec awarii bez utraty danych wynosi:}%
		{Nie}{1, dla 2 dysków pracujących w RAID0}%
		{Tak}{1, dla 3 dysków pracujących w RAID5}%
		{Tak}{1, dla 2 dysków pracujących w RAID1}%
		{Nie}{2, dla 3 dysków pracujących w RAID5}
		\item \question{RAID:}%
		{Tak}{jest stosowane w celu zwiększenia niezawodności}%
		{Nie}{wymaga minimum 3 dysków fizycznych do pracy}%
		{Tak}{jest stosowane w celu zwiększenia wydajności transmisji danych}%
		{Tak}{jest stosowane w celu powiększenia przestrzeni dostępnej jako jedna całość}
		\item \question{Mirrored volume w systemie Windows 2008 ma następujące właściwości:}%
		{Tak}{może chronić wolumen bootowalnego systemu operacyjnego Windows 2008}%
		{Nie}{do założenia wymaga 2 identycznych partycji na dyskach typu „basic disk”}%
		{Tak}{można go utworzyć na 2 dyskach}%
		{Nie}{wymaga zakupienia specjalnego kontrolera dysków}
		\item \question{Które z poniższych zdań na temat macierzy RAID5 są prawdziwe?}%
		{Tak}{RAID5 działa poprawnie do awarii więcej niż jednego dysku}%
		{Nie}{Macierz RAID5 wymaga minimum 4 dysków}%
		{Nie}{W n-dyskowej macierzy bity parzystości są na n-1 dyskach}%
		{Tak}{Macierz złożona z n jednakowych dysków ma objętość n-1 dysków}
		\item \question{Aby wykorzystać programowy RAID5 w systemie Windows 2008 Serwer należy posiadać komputer z zainstalowanymi}%
		{Nie}{trzema dyskami}%
		{Nie}{trzema dyskami oraz kontrolerem umożliwiającym systemowi Windows 2008 Server utworzenie programowej macierzy RAID5}
		{Tak}{czterema dyskami}%
		{Tak}{pięcioma dyskami}%
		\newpage
		\item \question{Dla których wolumenów prawdopodobieństwo utraty danych jest większe niż dla wolumenu prostego (simple volume):}%
		{Tak}{spanned volume}%
		{Tak}{striped volume}%
		{Nie}{RAID5}%
		{Nie}{mirrored volume}
		\item \question{Na ilu dyskach można założyć wolumen paskowany używając systemu operacyjnego Windows 2008?}%
		{Nie}{na 1}%
		{Tak}{na 2}%
		{Tak}{na 3}%
		{Tak}{na 4}
		\item \question{Zaznacz poprawne stwierdzenia dotyczące dysków podstawowych i dynamicznych w systemach Windows:}%
		{Nie}{Dyski podstawowe posiadają te same możliwości i funkcje co dyski dynamiczne jednak ich konfiguracja jest nieco trudniejsza}%
		{Nie}{Dyski dynamiczne dostępne są tylko w systemach windows z rodziny serwer}%
		{Tak}{Dyski podstawowe pozwalają na tworzenie podstawowych partycji, rozszerzonych partycji oraz dysków logicznych}%
		{Tak}{W niektórych wersjach systemu windows istnieje możliwość scalenia kilku oddzielnych dynamicznych dysków w jeden wolumen dynamiczny}
		\item \question{Na komputerze posiadającym 6 dysków zostanie zainstalowany system operacyjny Windows 2008 Server. Która konfiguracja pozwoli na pracę z najlepszym wykorzystaniem przestrzeni na dyskach zakładając, że nie można użyć macierzy sprzętowych?}%
		{Nie}{2 dyski spięte w mirror, 3 dyski spięte w RAID5}%
		{Tak}{2 dyski spięte w mirror, pozostałe 4 dyski spięte w wolumen paskowany}%
		{Nie}{wszystkie 6 dysków spiętych w RAID5}%
		{Nie}{utworzone 3 mirrory po 2 dyski każdy}
		\item \question{Na ilu dyskach można założyć wolumen paskowany używając systemu operacyjnego Windows 7?}%
		{Nie}{na 1}%
		{Tak}{na 2}%
		{Tak}{na 3}%
		{Tak}{na 5}
		\item \question{Na komputerze posiadającym 3 dyski zostanie zainstalowany system operacyjny Windows 2008 Server. Która konfiguracja pozwoli na pracę z najlepszym wykorzystaniem przestrzeni na dyskach zakładając, że nie można użyć macierzy sprzętowych?}%
		{Tak}{2 dyski spięte w mirror, jeden dysk bez zabezpieczeń}%
		{Nie}{3 dyski spięte w spanned volume}%
		{Nie}{wszystkie 3 dyski spięte w RAID5}%
		{Nie}{wszystkie dyski spięte w mirror}
		\item \question{Które konfiguracje RAID zwiększają wydajność (gdzie wzrost wydajności należy zrozumieć jako wzrost prędkości odczytu i zapisu)?}%
		{Tak}{RAID0}%
		{Tak}{RAID0+1}%
		{Tak}{RAID1+0}%
		{Nie}{RAID1}
		\item \question{W systemie Windows 7 na 5 dyskach za pomocą systemu operacyjnego został założony RAID5. Po pewnym czasie podczas pracy systemu 1 dysk uległ uszkodzeniu.}%
		{Nie}{odzyskiwanie danych będzie możliwe tylko z ostatniej archiwizacji}%
		{Nie}{jeśli uszkodzony dysk zostanie wymieniony na nowy to po ponownym uruchomieniu systemu dane zostaną automatycznie odzyskane}%
		{Nie}{danych nie będzie można odzyskać}%
		{Tak}{w systemie Windows 7 nie można użyć RAID5}
		{\small \emph{Uzasadnienie:} W systemie Windows 7 nie można założyć RAID5, gdyż taki poziom RAID jest dostępny dopiero w systemach serwerowych.}
		\item \question{Konfiguracja RAID0:}%
		{Tak}{Pojemność wszystkich połączonych dysków jest równa N*pojemność\_najmniejszego\_dysku, gdzie N to liczba połączonych dysków.}%
		{Tak}{Nie dostarcza żadnego zabezpieczenia danych.}%
		{Tak}{Znajduje idealne zastosowanie gdzie wydajność jest ważniejsza od bezpieczeństaw danych.}%
		{Nie}{Pojemność wszystkich połączonych dysków jest równa pojemności najmniejszego z nich.}
		\item \question{Jakie są dostępne typy dysków dynamicznych w systemie Windows 2003?}%
		{Tak}{Mirror}%
		{Tak}{Spanned Volume}%
		{Tak}{Stripped Volume}%
		{Tak}{Simple Volume}
		\item \question{W konfiguracji RAID1:}%
		{Tak}{Dane zapisywane są na obu dyskach równocześnie.}%
		{Nie}{Dane są zapisywane na kolejnych dyskach bit po bicie, tak jak w przypadku RAID2.}%
		{Tak}{Efektywna pojemność wynosi 50\% pojemności dysków.}%
		{Nie}{Wykorzystuje paskowanie dysków.}
		\item \question{Które z poniższych zdań opisują macierz RAID1 (mirroring)?}%
		{Nie}{RAID1 oferuje możliwość strippingu danych.}%
		{Nie}{Całkowita pojemność danych macierzy jest równa pojemności największego dysku.}%
		{Nie}{Pojemność macierzy jest równa pojemności najmniejszego dysku pomnożonego przez liczbę dysków.}%
		{Tak}{Odporność na awarię $ N-1 $ dysków w $ N $-dyskowej macierzy.}
		\newpage
		\item \question{W przypadku którego typu konfiguracji dysków istnieje możliwość odzyskania danych jeśli jeden z dysków macierzy ulegnie awarii?}%
		{Nie}{konfiguracja typu stripped volume}%
		{Tak}{konfiguracja typu RAID5}%
		{Tak}{konfiguracja typu mirror}%
		{Nie}{konfiguracja typu spanned volume}
		\item \question{Mirrored volume w systemie Windows 2008 ma następujące właściwości:}%
		{Tak}{może chronić wolumen z bootowalnym systemem operacyjnym Windows 2008.}%
		{Nie}{może obejmować więcej niż 2 dyski.}%
		{Nie}{całkowicie likwiduje ryzyko utraty danych.}%
		{Tak}{nie można go założyć na dyskach typu "basic disk".}
		\item \question{Który z typów RAID zapewni bezpieczeństwo przy awarii jednego dysku?}%
		{Tak}{RAID0+1}%
		{Nie}{RAID0}%
		{Tak}{RAID1}%
		{Tak}{RAID5}
		\item \question{Wskaż poprawną odpowiedź:}%
		{Tak}{Przestrzeń macierzy w RAID0 jest zależna od rozmiaru najmniejszego z użytych dysków.}%
		{Nie}{RAID0+1 i RAID1+0 udostępniają 100\% sumy pojemności wszystkich użytych dysków.}%
		{Nie}{RAID4 to macierz, której dane na dyskach są paskowane.}%
		{Tak}{Awaria dwóch dysków w RAID6 nie powoduje utraty danych.}
		\item \question{Programowy RAID5 w systemie Windows 2008 Server:}%
		{Nie}{można utworzyć już na 2 dyskach.}%
		{Tak}{można utworzyć na 4 dyskach.}%
		{Tak}{Zwiększa odporność systemu na awarie dysków.}%
		{Nie}{można założyć na dyskach typu "dynamic" lub basic.}
		\item \question{Jakie właściwości ma programowy RAID5 w systemie operacyjnym Windows 2008?}%
		{Tak}{można go założyć na 5 dyskach.}%
		{Nie}{umożliwia lepsze wykorzystanie przestrzeni na dyskach niż wolumen paskowany.}%
		{Nie}{zapewnia bezawaryjną pracę systemu.}%
		{Nie}{pozwala uniknąć fragmentacji systemu plików.}
		\item \question{Zaznacz zdania prawdziwe:}%
		{Nie}{RAID występuje wyłącznie sprzętowy.}%
		{Nie}{RAID występuje wyłącznie programowy.}%
		{Tak}{RAID występuje zarówno programowy jak i sprzętowy.}%
		{Nie}{Nie ma żadnej możliwości uruchomienia RAID w domowym komputerze PC.}
		\item \question{Które z podanych zdań są prawdziwe?}%
		{Tak}{RAID programowy pozwala na bezpośredni start systemu z macierzy dyskowej.}%
		{Nie}{RAID sprzętowy posiada wyższą wydajność od RAID programowego, gdyż przeliczaniem sum kontrolnych zajmuje się dedykowany kontroler.}%
		{Nie}{RAID programowy posiada większą kompatybilność z mniej popularnymi systemami operacyjnymi, gdyż wszystkie systemy operacyjne obsługują technologię RAID.}%
		{Tak}{RAID sprzętowy pozwala na bezpośredni start systemu z macierzy dyskowej.}
		\item \question{W systemie windows 2008 na 5 dyskach za pomocą systemu operacyjnego został założony RAID5 Po pewnym czasie podczas pracy systemu 2 dyski uległy uszkodzeniu.}
		{Nie}{jeśli uszkodzone dyski zostaną wymienione na nowe to po ponownym uruchomieniu systemu dane zostaną automatycznie odzyskane}
		{Nie}{odzyskiwanie danych będzie przezroczyste dla użytkowników jeśli dyski są typu hot swap}
		{Nie}{w systemie Windows 2008 nie można użyć RAID5}
		{Tak}{dane będzie można odzyskać tylko z archiwizacji, a nie z RAID5}
		{\small \emph{Uzasadnienie:} po awarii 2 dysków RAID5 traci dane.}
		\item \question{Jakie właściwości ma programowy RAID5 na systemie operacyjnym Windows 2008?}%
		{Tak}{można go założyć na pięciu dyskach}%
		{Nie}{umożliwia lepsze wykorzystanie przestrzeni na dyskach niż wolumen paskowany}%
		{Nie}{zapewnia bezawaryjną pracę systemu}%
		{Nie}{pozwala uniknąć fragmentacji systemu plików}
		\item \question{Konfiguracja RAID2:}%
		{Tak}{jest rozszerzeniem architektury RAID0}%
		{Tak}{dane są zapisywane na kolejnych dyskach macierzy bit po bicie}%
		{Nie}{cechuje się dużą wydajnością przy operacjach odczytu}%
		{Nie}{jest często stosowana w macierzach dyskowych}
		\item \question{Dyski typu podstawowego (ang. basic disks) pozwalają na:}%
		{Tak}{oznaczenie partycji jako aktywnej}%
		{Nie}{rozszerzenie woluminów prostych (ang. simple volume)}%
		{Tak}{tworzenie partycji podstawowej}%
		{Nie}{tworzenie woluminów RAID5}
		\item \question{Dla których wolumenów prawdopodobieństwo utraty danych jest mniejsze niż dla wolumenu łączonego (spanned volume):}%
		{Tak}{mirrored volume}%
		{Nie}{striped volume}%
		{Tak}{simple volume}%
		{Tak}{RAID5}
		\item \question{Jakie właściwości ma programowy RAID5 na systemie operacyjnym Windows 2008?}%
		{Nie}{zapewnia bezawaryjną pracę systemu}%
		{Tak}{chroni system przed awarią tylko jednego dysku}%
		{Nie}{pozwala uniknąć fragmentacji systemu plików}%
		{Nie}{umożliwia lepsze wykorzystanie przestrzeni na dyskach niż wolumen paskowany}%		
		
		%\item \question{}%
		%{Tak}{}%
		%{Nie}{}%
		%{}{}%
		%{}{}
		
	\end{enumerate}
	
	% --- Interpreter poleceń PowerShell --- %
	%\item \question{}%
%{Tak}{}%
%{Nie}{}%
%{}{}%
%{}{}

% !TeX spellcheck = pl_PL

% *************************************************************
% --- Sprawdzić poprawność Children oraz ChildItem
% *************************************************************
\newpage
\section{Interpreter poleceń PowerShell}
	\begin{enumerate}
		\item \question{Polecenie$>$ get-childitem C:\textbackslash test\textbackslash * -include *.txt -recurse $ \mid $ remove-item}%
		{Tak}{Znajduje i usuwa wszystkie pliki z rozszerzeniem .txt z folderu "C:\textbackslash test" i podfolderów.}%
		{Nie}{Znajduje i usuwa wszystkie pliki z rozszerzeniem .txt z folderu "C:\textbackslash test", bez podfolderów.}%
		{Nie}{Znajduje i wypisuje wszystkie pliki z rozszerzeniem .txt z folderu "C:\textbackslash test", bez podfolderów.}%
		{Nie}{Jest niepoprawne}
		\item \question{Które wersje systemu Windows NIE wpierają PowerShella?}%
		{Tak}{Windows 2000 SP4}%
		{Tak}{Windows 2000}%
		{Nie}{Windows Server 2008}%
		{Nie}{Windows 7}
		\item \question{Które polityki wykonywania skryptów w PowerShell zabraniają wykonywania skryptów pochodzących z lokalnego komputera, jeśli skrypty te nie są podpisane przez zaufanego wydawcę?}%
		{Tak}{Restricted}%
		{Tak}{AllSigned}%
		{Nie}{RemoteSigned}%
		{Nie}{Unrestricted}
		\item \question{Po wykonaniu poniższego skryptu w PowerShell\\
			\$przedmiot = "DSO" if (\$przedmiot -eq "DSO") \{"Dedykowane Systemy Operacyjne"\} elseif (\$przedmiot -eq "PK") \{"Programowanie Komputerów"\} else \{"Nieznany przedmiot"\}}%
		{Nie}{Na ekranie zostanie wyświetlony napis "Nieznany przedmiot".}%
		{Tak}{Wartość zmiennej \$przedmiot nie ulegnie zmianie.}%
		{Nie}{Na ekranie pojawi się komunikat o błędzie składniowym.}%
		{Nie}{Do zmiennej \$przedmiot zostanie przypisana wartość "Dedykowane Systemy Operacyjne".}
		\item \question{Aby zwrócić wszystkie obiekty w bieżącej lokalizacji nalezy użyć polecenia:}%
		{Tak}{Get-children}%
		{Nie}{Copy-item}%
		{Nie}{Get-content}%
		{Nie}{Get-process}
		\item \question{Polecenie "PS$>$ get-process d* $ \mid $ stop-process"}%
		{Tak}{poszczególne polecenia należą do tzw. poleceń Cmdlet. (należy do poleceń Cmdlet - inna odpowiedź) }%
		{Nie}{zatrzymuje wszystkie uruchomione procesy.}%
		{Nie}{zatrzymuje wszystkie procesy działające na partycji D.}%
		{Tak}{zatrzymuje wszystkie procesy których nazwa rozpoczyna się literą "d".}
		
		\newpage
		\item \question{Aby zwrócić wszystkie obiekty w bieżącej lokalizacji należy użyc polecenia:}%
		{Nie}{Get-process}%
		{Nie}{Copy-item}%
		{Nie}{Get-content}%
		{Tak}{Get-children}
		\item \question{Zaznacz poprawne przyporządkowania aliasów do Cmdletów}%
		{Nie}{taskkill -$>$ Kill-Process}%
		{Tak}{ls -$>$ Get-ChildItem}%
		{Tak}{help -$>$ Get-Help}%
		{Tak}{man -$>$ Get-Help}
		\item \question{Polecenie Get-EventLog w Windows PowerShell pozwala:}%
		{Nie}{Zapisywać informacje do dziennika zdarzeń.}%
		{Nie}{Takie polecenie nie istnieje.}%
		{Tak}{Pobierać wpisy z dziennika zdarzeń.}%
		{Nie}{Pobierać wpisy z pliku C:\textbackslash Var\textbackslash Log\textbackslash Messages.}
		\item \question{Polecenia natywne dla Windows PowerShell, które pozwalają na wykonywanie podstawowych operacji na obiektach w środowisku WPS to:}%
		{Nie}{Potoki (pipelines)}%
		{Tak}{Aplety poleceń (cmdlets)}%
		{Nie}{Aplety skryptowe (scriptlets)}%
		{Nie}{Komendy linii poleceń (line commands)}
		\item \question{Wskaż wszystkie poprawne zdania dotyczące interpretera Windows PowerShell:}%
		{Tak}{PowerShell jest oparty o .NET}%
		{Nie}{PowerShell nie udostępnia mechanizmów potoku.}%
		{Tak}{PowerShell pozwala ustawić różne polityki kontrolujące jakie skrypty można uruchomić.}%
		{Nie}{PowerShell jest kompatybilny z bashem.}
		\item \question{Polityka Restricted wykonywania plików:}%
		{Tak}{Jest domyślną polityką w środowisku PowerShell.}%
		{Nie}{Pozwala na uruchamianie skryptów z rozszerzeniem .ps1.}%
		{Nie}{Nie pozwala na wykonywanie komend w oknie interpretera.}%
		{Nie}{Pozwala na uruchamianie skryptów z rozszerzeniem .ps1xml.}
		\item \question{Które polecenie wypisze zawartość bieżącego katalogu z pominięciem plików o rozszerzeniu .exe?}%
		{Nie}{Dir *.exe}%
		{Tak}{gci -exclude *.exe}%
		{Tak}{Get-Childitem -exclude *.exe}%
		{Nie}{ls -include *.exe}
		
		\newpage
		\item \question{Wskaż poprawne polecenia PowerShell usuwające z bieżącego katalogu pliki większe niż 2kB:}%
		{Nie}{Get-Childitem $ \mid $  Where-Object ( \$\_.length $>$ 2kB ) $ \mid $ Remove-Item}%
		{Nie}{Get-Childitem $ \mid $ Remove-Item $ \mid $ Where ( \$\_.length $>$ 2kB )}%
		{Tak}{Get-Childitem $ \mid $ Where-Object ( \$\_.length -gt 2kB ) $ \mid $ Remove-Item}%
		{Tak}{ls $ \mid $ where-object \{ \$\_.length -gt 2kB \} $ \mid $ rm}
		\item \question{Polecenie\\ "PS$ > $ get-process $ \mid $ where-object { \$\_.WS -gt 300MB } $ \mid $ stop-process"\\ wydane w interpreterze Windows PowerShell:}%
		{Nie}{Listuje procesy, które zużywają więcej niż 300 MB.}%
		{Nie}{Szuka procesu, który zużywa więcej niż 300 MB i wyświetla nazwę.}%
		{Tak}{Szuka procesu, który zużywa więcej niż 300 MB i zatrzymuje go.}%
		{Nie}{Szuka procesu, który zużywa mniej niż 300 MB i zatrzymuje go.}
		\item \question{Która z wersji systemu Windows obsługuje interpreter PowerShell?}%
		{Tak}{Windows Vista}%
		{Tak}{Windows 7}%
		{Tak}{Windows XP SP2/SP3}%
		{Nie}{Windows 95}
		\item \question{Polecenie Set-Location w Cmdlets (PowerShell) ma swój odpowiednik w interpreterze komend cmd.exe. Jest to:}%
		{Tak}{chdir}%
		{Nie}{set}%
		{Nie}{sloc}%
		{Tak}{cd}
		\item \question{Które z poleceń są poprawnymi podstawowymi aliasami w Windows PowerShell, służącymi do czyszczenia ekranu?}%\textsl{}
		{Nie}{Clear-Console}%
		{Nie}{Clear-Host}%
		{Tak}{clear}%
		{Tak}{cls}
		\item \question{W celu zatrzymania procesów zużywających więcej niż 100MB pamięci RAM należy użyć polecenia:}%
		{Nie}{PS$ > $ stop-process $ \mid $ where-object \{ \$\_.WS -gt 100MB \}}%
		{Nie}{PS$ > $ stop-process \$Memory -gt 100MB}%
		{Nie}{PS$ > $ get-process $ \mid $ where-object \{ \$Memory -gt 100MB \} $ \mid $ stop-process}%
		{Tak}{PS$ > $ get-process $ \mid $ where-object \{ \$\_.WS -gt 100MB \} $ \mid $ stop-process}
		
		\newpage
		\item \question{Zaznacz poprawne zdania dotyczące powłoski PowerShell:}%
		{Tak}{Wszystkie zmienne są obiektami .NET.}%
		{Tak}{Do zmiennych odwołuje się używając znaku \$.}%
		{Nie}{Część zmiennych jest obiektami .NET.}%
		{Nie}{Do zmiennych odwołuje się używając znaku \#.}
		\item \question{Za pomocą polecenia:\\Get-Childitem C:\textbackslash Work\textbackslash  -Recurse -Force $ \mid $ Measure-Object -property length -sum\\(Komentarz: polecenie measure-object służy do generowania statystyk)}%
		{Tak}{Znajdziemy liczbę plików i ich całkowity rozmiar w folderze C:\textbackslash Work oraz w podfolderach.}%
		{Nie}{Wypiszemy zawartość folderu C:\textbackslash Work.}%
		{Nie}{Wypiszemy największy plik z folderu C:\textbackslash Work.}%
		{Nie}{Jest to niepoprawna składnia.}
		\item \question{Aby usunąć wszystkie pliki z katalogu c:\textbackslash temp\textbackslash o rozszerzeniu .xls w Windows PowerShell należy użyć polecenia:}%
		{Tak}{remove-item c:\textbackslash temp\textbackslash *.xls}%
		{Tak}{get-childitem c:\textbackslash temp\textbackslash *.xls $ \mid $ foreach-object \{ remove-item \$\_.fullname \}}%
		{Nie}{remove-item c:\textbackslash temp\textbackslash * -exclude *.xls}%
		{Nie}{remove-file c:\textbackslash temp\textbackslash * -extension xls}
		\item \question{Polecenie:\\PS$ > $ get-childitem C:\textbackslash test\textbackslash * -include *.txt -recurse $ \mid $ remove-item }%
		{Tak}{Znajduje i usuwa wszystkie pliki z rozszerzeniem .txt z folderu "C:\textbackslash test" i podfolderów}%
		{Nie}{Znajduje i usuwa wszystkie pliki z rozszerzeniem .txt z folderu "C:\textbackslash test", bez podfolderów}%
		{Nie}{Znajduje i wypisuje wszystkie pliki z rozszerzeniem .txt z folderu "C:\textbackslash test", bez podfolderów}%
		{Nie}{Jest niepoprawne.}
		\item \question{Jakie rozszerzenia mogą mieć skrypty PowerShell?}%
		{Nie}{.wps}%
		{Nie}{.shl}%
		{Nie}{.cmd}%
		{Tak}{.ps1}
		\item \question{Której z niżej wymienionych polityk uruchamiania skryptów są dostępne w powerShell systemu Windows?}%
		{Nie}{NoneAllowed - nie pozwala na uruchamianie żadnych skryptów.}%
		{Tak}{AllSigned - możliwość uruchomienia tylko podpisanych skryptów.}%
		{Tak}{RemoteSigned - możliwość uruchamiania skryptów lokalnych oraz podpisanych pochodzących z Internetu.}%
		{Tak}{Unrestricted - pozwala na uruchamianie wszystkich skryptów.}
		
		\newpage
		\item \question{Czym charakteryzują się komendy (tzw. cmdlety) w PowerShell?}%
		{Tak}{Zazwyczaj zwracają obiekty.}%
		{Nie}{Nie mogą mieć zdefiniowanych kilku aliasów jednocześnie.}%
		{Nie}{Mają nazwy postaci "rzeczownik-czasownik"}%
		{Tak}{Mają nazwy postaci "czasownik-rzeczownik"}
		\item \question{Aby uzyskać pomoc na temat poleceń w Windows PowerShell należy użyć polecenia:}%
		{Nie}{please}%
		{Tak}{help}%
		{Nie}{Oh genie}%
		{Tak}{Get-Help}
		\item \question{Aby sprawdzić czy jakiś katalog już istnieje w Windows PowerShell można skorzystac z poleceń:}%
		{Nie}{remove-item}%
		{Tak}{test-path}%
		{Nie}{path}%
		{Nie}{new-item}
		\item \question{Wskaż wszystkie prawdziwe zdania dotyczące interpretera Windows PowerShell:}%
		{Tak}{Polecenie ls jest aliasem polecenia Get-Children.}%
		{Nie}{PowerShell nie posiada modułów i przystawek pozwalających na rozszerzanie powłoki poprzez dodawanie nowych cmdletów.}%
		{Nie}{W systemie operacyjnym Windows XP SP2 domyślnie zainstalowaną wersją PowerShella jest wersja "PowerShell v2"}%
		{Tak}{PowerShell pozwala na przetwarzanie potokowe, które pozwala na przekazywanie obiektu z jednego cmdletu do drugiego, bez potrzeby korzystania z parsowania tekstu czy zmiany formatowania.}
		\item \question{Polecenie:
			"new-item c:\textbackslash temp\textbackslash test -type directory"\\
			spowoduje:}%
		{Nie}{Utworzenie katalogu directory w katalogu c:\textbackslash temp\textbackslash test}%
		{Nie}{Sprawdzi istnienie katalogu test w katalogu c:\textbackslash temp}%
		{Tak}{Utworzenie katalogu test w katalogu c:\textbackslash temp}%
		{Nie}{Sprawdzi czy "test" w katalogu c:\textbackslash temp jest katalogiem}
		\item \question{Które wersje systemu Windows NIE wspierają PowerShella?}%
		{Nie}{Windows Vista}%
		{Tak}{Windows 2000}%
		{Nie}{Windows XP SP2}%
		{Nie}{Windows 7}
		
		\newpage
		\item \question{Które wersje systemu Windows NIE wspierają PowerShella?}%
		{Tak}{Windows 2000}%
		{Tak}{Windows 2000 SP4}
		{Nie}{Windows Server 2008}%
		{Nie}{Windows 7}%
		\item \question{Wskaż wszystkie prawdziwe zdania dotyczące interpretera Windows PowerShell:}%
		{Tak}{Wszystkie zmienne są obiektami .NET.}%
		{Tak}{Aby skopiować plik należy wpisać polecenie "Copy-item lokalizacja1 lokalizacja2"}%
		{Nie}{Aby skopiować plik należy wpisać polecenie "Set-Location lokalizacja1 lokalizacja2"}%
		{Tak}{PowerShell jest elementem pakietu Windows Management Framework.}
		\item \question{W Windows PowerShell poprawnie stworzona pętla to:}%
		{Tak}{ \$a = 1 do \{ \$a; \$a++ \} while (\$a -lt 10) }%
		{Nie}{ \$a = 10 do \{ \$a; \$a$ -- $ \} while (\$a -lt 3) }%
		{Tak}{ for (\$a = 1; \$a -le 10; \$a++) \{ \$a \} }%
		{Nie}{ foreach ( \$i in get-child c:\textbackslash scripts ) \{\$i.extended\} }
		\item \question{Co należy wstawić w miejsce znaków zapytania, aby poniższy skrypt PowerShella wyświetlał nazwę procesu w danej chwili najbardziej obciążającego procesor?\\
			\$ps = get-process\\
			\$max = \$ps[0]\\
			foreach (\$p in \$ps )\\
			\{\\
				if ( ??? )\\
				\{ \$max = \$p \}
			\}\\
			\$max.processname
			}%
		{Nie}{ \$p $ > $ \$max }%
		{Tak}{ \$p.cpu -gt \$max.cpu }%
		{Nie}{Brak odpowiedzi w źródle.}%
		{Nie}{Brak odpowiedzi w źródle.}
		\item \question{Aby wyświetlić wszystkie pliki o rozszerzeniu .txt znajdujące się w obecnym katalogu można użyć polecenia:}%
		{Tak}{Get-ChildItem *.* -include *.txt}%
		{Nie}{Get-ChildItem -extension *.txt}%
		{Nie}{Get-ChildItem -exclude *.txt}%
		{Tak}{Get-ChildItem $ \mid $ Where-Object \{\$\_.Attributes -ne "Directory" -and \$\_.Extension -eq ".txt"\}}
		\item \question{Zaznacz prawidłowe komendy ustawiające aktualną lokalizację na „C:\textbackslash”:}%
		{Tak}{Set-Location c:\textbackslash}%
		{Tak}{chdir c:\textbackslash}%
		{Tak}{cd c\textbackslash}%
		{Nie}{goto c:\textbackslash}
		
		\newpage
		\item \question{Zaznacz wszystkie prawidłowe odpowiedzi opisujące Windows PowerShell (WPS):}%
		{Nie}{WPS to narzędzie open source do zarządzania systemami Windows spod konsoli linuxowej}%
		{Tak}{WPS zapewnia dostęp do obiektów COM}%
		{Tak}{WPS to środowisko oparte na platformie .NET}%
		{Tak}{WPS to środowisko do automatyzowania zadań administracyjnych przy użyciu skryptów}
		\item \question{Polecenie\\dir -exclude *.zip -name -recurse -force}%
		{Tak}{Wyświetli nazwy wszystkich plików znajdujących się w danym katalogu, wraz z plikami ze wszystkich podfolderów i ich podfolderów, wraz z plikami ukrytymi i bez dostępu do nich, bez plików z rozszerzeniem zip}%
		{Nie}{Wyświetli nazwy wszystkich plików, z pominięciem plików z rozszerzeniem zip, znajdujących się tylko w danym katalogu, wraz z plikami ukrytymi i bez dostępu do nich.}%
		{Nie}{Wyświetli nazwy wszystkich plików znajdujących się tylko w danym katalogu, wraz z plikami ukrytymi i bez dostępu do nich.}%
		{Nie}{Wyświetli nazwy wszystkich plików z rozszerzeniem zip znajdujących się w danym katalogu, wraz ze wszystkimi podfolderami, wraz z plikami ukrytymi i bez dostępu do nich.}
		\item \question{Wskaż polecenia działające w powłoce bash oraz powershell}%
		{Tak}{man}%
		{Tak}{cp}%
		{Tak}{cd}%
		{Nie}{gps}
		\item \question{Zmienne w interpreterze PowerShell:}%
		{Tak}{nie muszą być deklarowane}%
		{Nie}{wymagają określenia typu}%
		{Tak}{mogą mieć różne typy}%
		{Tak}{są obiektami .NET}
		\item \question{Polecenie PoweShell:\\„PS$ > $ get-process a* $ \mid $ stop-process”}%
		{Nie}{Dotyczy wszystkich procesów (a* = all)}%
		{Nie}{Jest poleceniem błędnym – nie wykona się}%
		{Tak}{Zatrzyma procesy, których lista jest pobierana za pomocą polecenia get-process a*}%
		{Tak}{Dotyczy tylko procesów, których nazwa zaczyna się na literę „a”}
		\item \question{W interpreterze PowerShell polecenie Get-Process:}%
		{Tak}{Pozwala wypisać wszystkie aktualnie uruchomione procesy}%
		{Nie}{Pozwala zmieniać priorytet procesu}%
		{Nie}{Pozwala zmieniać właściciela procesu na aktualnie zalogowanego użytkownika}%
		{Nie}{Przekierowywuje wynik działania procesu (standardowe wyjście) do pliku}
		
		\newpage
		\item \question{Interpreter Windows PowerShell:}%
		{Nie}{w systemie Windows 7 (lub Windows Server 2008) wymaga wcześniejszej instalacji}%
		{Tak}{Jest zintegrowany z .NET Framework}%
		{Tak}{Dostarcza środowisko do wykonywania zadań administracyjnych wykonywanych poleceniami cmdlets}%
		{Tak}{Wynikiem polecenia w interpreterze jest ciąg obiektów określonego typu}
		\item \question{Zanzacz wszystkie prawidłowe sformułowania dotyczące powłoki PowerShell:}%
		{Tak}{Dzięki operatorowi $ \mid $ (tzw. pipe) można przekierować wyjście jednego polecenia na wejście drugiego, np. get-process $ \mid $ stop-process}%
		{Tak}{Wszystkie zmienne są obiektami .NET}%
		{Nie}{Skrypty pisane dla linuksowego interpretera Bash mogą być uruchamiane w interpreterze PowerShell}%
		{Nie}{Polecenia PowerShell mają ściśle określone nazwy, do których nie można tworzyć aliasów.}
		\item \question{Które z poniższych par słów przedstawiają pewien cmdlet oraz jego alias w Windows PowerShell?}%
		{Tak}{Set-Location, cd}%
		{Tak}{Get-Help, man}%
		{Nie}{Remove-File, rm}%
		{Tak}{Remove-Item, del}
		\item \question{Liczby od 1 do 5 wypisze następujący skrypt:}%
		{Tak}{\$i = 1\\
			do \{\\
				Write-Host \$i\\
				\$i++\\
			\}\\
			while (\$i -le 5)}%
		{Tak}{\$i = 1\\
			do \{\\
				echo \$i\\
				\$i++\\
			\}\\
			while (\$i -le 5)}%
		{Nie}{\$i = 1\\
			do \{\\
				echo i\\
				i++\\
			\}\\
			while (\$i -le 5)}%
		{Nie}{\$i = 1\\
			do \{\\
				print \$i\\
				i++\\
			\}\\
			while (\$i -le 5)}
		
		\newpage
		\item \question{Które polecenia są poprawne i wyświetlają, posortowaną wg. pewnej kolumny, zawartośd bieżącego katalogu?}%
		{Tak}{ls $ \mid $ Sort-Object Name}%
		{Nie}{ls $ \mid $ Sort-Name}%
		{Tak}{ls $ \mid $ Sort-Object Length}%
		{Nie}{ls $ \mid $ Sort(Length)}
		\item \question{Polecenie:\\"get-childitem C:\textbackslash* -include *.txt"\\wydane w Windows PowerShell:}%
		{Tak}{wyświetli nazwy wszystkich plików o rozszerzeniu ".txt" znajdujących się w ścieżce C:\textbackslash}%
		{Nie}{wyświetli nazwy wszystkich plików o rozszerzeniu ".txt" znajdujących się w ścieżce C:\textbackslash i jej podkatalogach}%
		{Nie}{wyświetli tylko nazwy wszystkich plików o rozszerzeniu ".txt" znajdujących się w ścieżce C:\textbackslash}%
		{Tak}{wyświetli m.in. nazwę i czas ostatniego czas ostatniego zapisu wszystkich plików o rozszerzeniu ".txt" znajdujących się w ścieżce C:\textbackslash}
		\item \question{Która z wersji systemu Windows obsługuje interpreter Windows PowerShell?}%
		{Nie}{Windows 98}%
		{Tak}{Windows XP}%
		{Tak}{Windows Vista}%
		{Tak}{Windows 7}
		\item \question{PS E:\textbackslash test$ > $ ls\\
			Directory: E:\textbackslash test\\
			Mode   LastWriteTime Length Name\\
			$ ----\;\;\;\;\;\;-------------\;\;\;------\;\;\;---- $\\
			$ -a--- $  2012-06-02    16:12  0 a.xyz\\
			$ -a--- $  2012-06-02    16:12  0 b.xyz\\
			$ -a--- $  2012-06-02    16:12  0 c.xyz\\
			$ -a--- $  2012-06-02    16:12  0 d.xyy\\
			$ -a--- $  2012-06-02    16:12  0 e.xxy\\
			PS E:\textbackslash test$ > $ get-childitem C:\textbackslash test\textbackslash * -include *.xyz -recurse $ \mid $ remove-item\\
			Zaznacz możliwe do otrzymania wyniki działania komendy ls z dowolnymi parametrami po wykonaniu powyższej komendy:\\
			\textbf{(\emph{Zbieracz Forczu}: te kreski są przerywane, zwykłe myślniki)}}%
		{Tak}{Directory: E:\textbackslash test\\
			Mode  LastWriteTime Length Name\\
			----  ------------- ------ ----\\
			-a--- 2012-06-02    16:12  0 d.xyy\\
			-a--- 2012-06-02    16:12  0 e.xyy}%
		{Tak}{Directory: E:\textbackslash test\\
			Mode  LastWriteTime Length Name\\
			----  ------------- ------ ----\\
			-a--- 2012-06-02    16:12  0 e.xxy}%
		{Tak}{d.xyy\\
			e.xxy\\
			PS E:\textbackslash test$ > $}%
		{Tak}{Directory: E:\textbackslash test\textbackslash }
		
		\newpage
		\item \question{Zdania prawdziwe, opisujące zmienne PowerShell, to:}%
		{Nie}{Zmienne muszą mieć zdefiniowany typ}%
		{Tak}{Wszystkie zmienne są obiektami .NET}%
		{Tak}{Nie muszą być deklarowane}%
		{Nie}{Wszystkie zmienne są globalne}
		\item \question{Które z podanych przykładów pętli są poprawne w PowerShell?}%
		{Tak}{foreach (\$i in get-childitem c:\textbackslash scripts) \{\$i.extension\}}%
		{Tak}{for(\$zm = 1; \$zm -le 10; \$zm++) \{\$zm\}}%
		{Nie}{for(a = 1; i $ < $ 10; ++i) \{i\}}%
		{Nie}{while(\$i -lt 10) \{\$i\}}
		\item \question{Które z przytoczonych niżej cech odnoszą się do powłoki PowerShell?}%
		{Nie}{Korzystanie z pętli for, while i until jest niedozwolone.}%
		{Tak}{Odwoływanie się do zmiennych jest możliwa poprzez użycie znaku \$}%
		{Nie}{Zmienne nie mogą mieć zakresów widoczności}%
		{Tak}{Wszystkie zmienne są obiektami .NET}
		\item \question{Które polecenie powłoki PowerShell wyświetli listę uruchomionych usług?}%
		{Tak}{Get-Service $ \mid $ Where-Object \{\$\_.name -eq "running"\}}%
		{Nie}{Get-Service $ \mid $ Where-Object \{\$\_.name -eq "SysMain"\}}%
		{Nie}{Get-Service $ \mid $ Where-Object \{\$\_.name -eq "stopped"\}}%
		{Nie}{Get-Process $ \mid $ Where-Object \{\$\_.name -eq "running"\}}
		\item \question{Aliasami polecenia Set-Location w Windows Powershell są:}%
		{Tak}{sl}%
		{Tak}{cd}%
		{Tak}{chdir}%
		{Nie}{setloc}
		\item \question{\$a = 5\\
			If (\$a -eq 5)\\
			\{”Piątka ”\}\\
			elseif (\$a -lt 6)\\
			\{”mniejsza od szóstki”\}\\
			If (\$a -gt 3)\\
			\{”większa od trójki”\}\\
			Po wykonaniu tego kodu w oknie Windows Powershell:}%
		{Nie}{Zostanie wypisany tekst „Piątka mniejsza od szóstki”}%
		{Nie}{Zostanie wypisany tekst „Piątka mniejsza od szóstki większa od trójki”}%
		{Tak}{Zostanie wypisany tekst „Piątka większa od trójki”}%
		{Nie}{Zostanie wypisany komunikat o błędzie w kodzie programu}
		
		\newpage
		\item \question{Które z wymienionych opisów dotyczy PowerShella:}%
		{Tak}{ułatwia zadania administracyjne}%
		{Nie}{jest zgodny linuksową powłoką shell}%
		{Nie}{może być wykorzystywany tylko przez administratorów}%
		{Tak}{może być wykorzystywany przez wszystkich użytkowników}
		\item \question{Wskaż prawdziwe zdania. Zdania dotyczą polityki uruchamiania skryptów w konsoli PowerShell.}%
		{Tak}{Polityka Unrestricted umożliwia uruchamianie niepodpisanych skryptów.}%
		{Nie}{Polityka Restricted umożliwia uruchomienie tylko tych skryptów, które pochodzą z lokalnego komputera.}%
		{Nie}{Polityka AllSigned jest polityką domyślną.}%
		{Tak}{Polityka AllSigned umożliwia uruchamianie skryptów które zostały podpisane przez zaufanego wydawcę lub pochodzą z komputera lokalnego.}
		\item \question{Wskaż prawdziwe zdania. Zdania dotyczą uruchamiania skryptów w konsoli PowerShell.}%
		{Tak}{W konsoli PowerShell nie jest możliwe uruchomienie skryptu bez podania jego pełnej ścieżki.}%
		{Nie}{W konsoli PowerShell jest możliwe uruchomienie skryptu bez podania jego pełnej ścieżki, jednak wymaga to ustawienia odpowiedniej polityki uruchamiania skryptów.}%
		{Nie}{W konsoli PowerShell jest możliwe uruchomienie jakiegokolwiek skryptu bez podania jego pełnej ścieżki pod warunkiem, że bieżącą ścieżką będzie folder zawierający skrypt oraz polityka uruchamiania skryptów jest ustawiona na Unrestricted.}%
		{Nie}{W konsoli PowerShell jest możliwe uruchomienie skryptu bez podawania jego pełnej ścieżki w przypadku gdy skrypt ten pochodzi z komputera lokalnego.}
		\item \question{Windows PowerShell:}%
		{Tak}{Jest zintegrowany z .NET Framework}%
		{Nie}{Dostępny jest dla systemu Windows 2000}%
		{Tak}{Jest interpreterem poleceń}%
		{Nie}{Zwraca w wyniku każdego polecenia zmienną typu string}
		\item \question{Polecenie:\\"get-childitem C:\textbackslash Kolokwium\textbackslash Main\textbackslash * -include *.kol -recurse $ \mid $ remove-item"\\w Windows PowerShell:}%
		{Tak}{Znajduje i usuwa wszystkie pliki z rozszerzeniem "kol" z folderu "C:\textbackslash Kolokwium\textbackslash Main" i jego podfolderów}%
		{Nie}{Znajduje i usuwa wszystkie pliki z rozszerzeniem "kol" z folderu nadrzędnego do "C:\textbackslash Kolokwium\textbackslash Main", tzn. "C:\textbackslash Kolokwium"}%
		{Nie}{Znajduje i usuwa wszystkie pliki z rozszerzeniem "kol" wyłącznie z folderu "C:\textbackslash Kolokwium\textbackslash Main"}%
		{Nie}{Żadna z odpowiedzi nie jest prawidłowa}
		
		\newpage
		\item \question{Wskaż poprawne zdania dotyczące zmiennych w Windows PowerShell:}%
		{Tak}{Wszystkie zmienne są obiektami .NET.}%
		{Nie}{Zmienne muszą mieć nadany typ.}%
		{Nie}{Wartość do zmiennej przypisuje operator „:=”}%
		{Tak}{Zmienne mogą mieć zakres widoczności.}
		\item \question{Które z poniższych skryptów PowerShella wydrukują listę nazw wszystkich plików o rozszerzeniu .txt w aktualnie wybranym katalogu?}%
		{Tak}{get-childitem $ \mid $ where-object \{\$\_.extension -eq ".txt"\} $ \mid $ format-table Name}%
		{Nie}{get-childitem $ \mid $ format-table Name $ \mid $ where-object \{\$\_.extension -eq ".txt"\}}%
		{Tak}{foreach(\$a in get-childitem) \{\\
				if(\$a.extension -eq ".txt") \{\\
					\$a.Name\\
				\}\\
			\}}%
		{Tak}{get-childitem $ \mid $ foreach \{if(\$\_.extension -eq ".txt")\{\$\_.Name\}\}}
		\item \question{Zamiennikiem poleceo dir i ls w PowerShell jest polecenie:}%
		{Tak}{Get-ChildItem}%
		{Nie}{Get-Content}%
		{Nie}{Tee-Object}%
		{Nie}{Set-Variable}
		\item \question{Co charakteryzuje PowerShell:}%
		{Tak}{Nie rozróżnia wielkości liter w komendach}%
		{Nie}{Każdą linię w pliku .ps należy zakończyć średnikiem}%
		{Tak}{Zmienne oznacza się znakiem dolara '\$'}%
		{Tak}{Istnieje różnica między pojedynczym a podwójnym cudzysłowem}
		\item \question{Które zdania o PowerShel są FAŁSZYWE:}%
		{Nie}{jest zintegrowany z .NET Framework}%
		{Tak}{GetChildItem zwraca wszystkie obiekty jakie zawierają dzieci bieżącej lokalizacji}%
		{Nie}{Zmienne są obiektami}%
		{Tak}{Do zmiennych odwołujemy się znakiem \%}
		\item \question{Instrukcja w PowerShel która zatrzymuje (ang.kill) procesy zaczynające się na literę Y to:}%
		{Tak}{get-process Y* $ \mid $ stop-process}%
		{Nie}{get-process Y* $ \mid $ kill-proces}%
		{Tak}{ps Y* $ \mid $ stop-process}%
		{Tak}{ps Y* $ \mid $ kill-process}
		
		\newpage
		\item \question{Użytkownik korzysta z Powershella w środowisku Windows i znajduje się w lokalizacji C:\textbackslash MyScripts$ > $ po wywołaniu komendy ls, okazało się, że w tym folderze znajduje się plik script.ps1. Użytkownik zamierzając go uruchomić, zmienił aktualna politykę wykonywania skryptów z Restricted na RemoteSigned. Które z poniższych komend uruchomią w/w skrypt?}%
		{Nie}{run script.ps1}%
		{Nie}{script.ps1}%
		{Tak}{C:\textbackslash MyScripts\textbackslash script.ps1}%
		{Tak}{.\textbackslash script.ps1}
		\item \question{Co się stanie po wywołaniu komendy:\\get-process pow* $ \mid $ stop-process}%
		{Tak}{Zostaną zatrzymane wszystkie procesy, których nazwa zaczyna się od ciągu znaków "pow"}%
		{Tak}{Powershell zostanie wyłączony}%
		{Nie}{Zostaną zatrzymane wszystkie procesy, których nazwa zawiera ciąg znaków "pow"}%
		{Nie}{Nic się nie stanie}
		\item \question{Wskaż poprawne polecenia PowerShell usuwające z bieżącego katalogu pliki większe niż 2kB:}%
		{Tak}{Get-Childitem $ \mid $ Where-Object \{ \$\_.length -gt 2kB \} $ \mid $ Remove-Item}%
		{Nie}{Get-Childitem $ \mid $ Where-Object ( \$\_.length $ > $ 2kB ) $ \mid $ Remove-Item}%
		{Nie}{Get-Childitem $ \mid $ Remove-Item $ \mid $ Where (\$length $ > $ 2kB)}%
		{Tak}{ls $ \mid $ where-object \{ \$\_.length -gt 2kB \} $ \mid $ rm}
		\item \question{Po wykonaniu w konsoli PowerShell polecenia Get-ExecutionPolicy otrzymano rezultat "Restricted". Oznacza to, że użytkownik:}%
		{Tak}{Nie może uruchamiać żadnych skryptów, a jedynie osobne komendy}%
		{Nie}{Może uruchamiać niepodpisane skrypty pochodzące z lokalnego komputera}%
		{Nie}{Może uruchamiać podpisane skrypty pobrane z Internetu}%
		{Nie}{Może uruchamiać niepodpisane skrypty pobrane z Internetu}
		\item \question{W PowerShell polecenie Get-Process:}%
		{Tak}{Wylistuje wszystkie aktualnie uruchomione procesy}%
		{Nie}{Zmieni priorytet procesu}%
		{Nie}{Zmieni właściciela procesu na aktualnie zalogowanego użytkownika}%
		{Nie}{Przekierowuje wynik działania procesu (standardowe wyjście) do pliku}
		\item \question{W PowerShell polityka bezpieczeostwa RemoteSigned zezwala na:}%
		{Nie}{Wykonywanie dowolnych skryptów.}%
		{Tak}{Uruchamianie skryptów podpisanych przez zaufanego wydawcę.}%
		{Tak}{Uruchamianie niepodpisanych skryptów, które powstały na lokalnym komputerze.}%
		{Nie}{Korzystanie jedynie z pojedynczych komend, bez możliwości uruchamiania skryptów.}
		
		\newpage
		\item \question{Prawidłowa postać pętli for w PowerShell to:}%
		{Nie}{for (i = 1, i -le 10, i++) \{ \}}%
		{Nie}{for (i = 1; i -le 10; i++) \{ \}}%
		{Nie}{for (\$i = 1; \$i $ < $= 10; i++) \{ \}}%
		{Tak}{for (\$i = 1; \$i -le 10; \$i++) \{ \}}
		\item \question{Jaki jest rezultat polecenia interpretera PowerShell:\\
			get-childitem C:\textbackslash Work\textbackslash -recurse $ \mid $ get-acl $ \mid $ where \{ \$\_.Owner -match "Maniek"\}}%
		{Tak}{Wypisze wszystkie pliki, których właścicielem jest Maniek z folderu C:\textbackslash Work oraz podfolderów}%
		{Nie}{Wypisze tylko pliki z folderu C:\textbackslash Work, których właścicielem jest Maniek.}%
		{Nie}{Wypisze wszystkie pliki z dysku C:, których właścicielem jest Maniek.}%
		{Nie}{Wypisze pliki, które nie należą do użytkownika Maniek, z folderu C:\textbackslash Work}
		\item \question{Co otrzymamy po wykonaniu następującej komendy w interpreterze PowerShell:\\
			PS C:\textbackslash $ > $ Get-ChildItem $ \mid $ where \{ !\$\_.PslsContainer  \} $ \mid $ Select-Object Name}%
		{Nie}{Tylko nazwy folderów jakie znajdują się w lokalizacji C:\textbackslash}%
		{Nie}{Tylko nazwy folderów i podfolderów jakie znajdują się w lokalizacji C:\textbackslash}%
		{Tak}{Tylko nazwy plików jakie znajdują się w lokalizacji C:\textbackslash}%
		{Nie}{Dokładny opis folderów, czyli m.in. nazwy i uprawnienia jakie znajdują się w lokalizacji C:\textbackslash}
		
		%\item \question{}%
		%{Tak}{}%
		%{Nie}{}%
		%{}{}%
		%{}{}
		
	\end{enumerate}
	
	% --- Windows API ---------------------- %
	

% Gdzieniegdzie w pytaniach są oznaczenia W12-xx i Ox. Cholera wie co to, może jakieś identyfikatory w bazie danych pytań. Nie przepisywałem ich. %

\newpage
\section{Windows API}

\begin{enumerate}
	
	\item \question{Do funkcji Windows APi należą:}
	{Tak}{CreateWindowsEx}
	{Nie}{strcmp}
	{Tak}{ShowWindow}
	{Nie}{atoi}
	
	\item \question{Kiedy musi być zarejestrowana klasa okna w Windows API}
	{Nie}{klasa okna może być zarejestrowana zarówno przed jak i po utworzeniu okna}
	{Tak}{przed utworzeniem okna}
	{Nie}{po utworzeniu okna}
	{Nie}{klasa okna nie jest rejestrowana w Window API}
	
	\item \question{HWND:}
	{Nie}{Jest strukturą przechowującą wskaźniki do poszczególnych okien aplikacji}
	{Nie}{Jest wskaźnikiem na funkcję obsługującą komunikaty napływające do okna aplikacji}
	{Tak}{Jest uchwytem okna aplikacji}
	{Nie}{Jest funkcją pozwalającą na zdefiniowanie głównego okna aplikacji}
	
	\item \question{Aby wyświetlić krótki komunikat dla użytkownika przy użyciu okna modalnego można użyć funkcji}
	{Nie}{ShowDialog(...)}
	{Nie}{MsgBox(...)}
	{Tak}{MessageBox(...)}
	{Nie}{ShowModDialog(...)}
	
	\item \question{Kod programów pisanych z bezpośrednim wykorzystaniem Win32API musi zawierać:}
	{Nie}{Instrukcję $\sharp$include}
	{Nie}{Wywołanie funkcji CreateWindowEx(...)}
	{Tak}{Funkcję WinMain}
	{Nie}{Funkcję WINAPI}
	
	\item \question{Windows API pozwala na:}
	{Tak}{komunikację sieciową}
	{Tak}{ostęp do systemu plików}
	{Tak}{tworzenie interfejsu graficznego}
	{Tak}{dostęp do rejestrów systemu}

	\item \question{MDi w API jest skrótem od:}
	{Nie}{Media Download Interface}
	{Nie}{Mass Data Interface}
	{Tak}{Multiple Data Interface}
	{Nie}{Multicolor Data Interface}

	\item \question{UpdateWindow:}
	{Tak}{Jest funkcją wysyłającą komunikat do okna aplikacji informującym go o potrzebie przerysowania}
	{Nie}{Jest domyślną funkcją obsługującą przerysowanie okna lub jego fragmentu}
	{Nie}{Jest komunikatem wysyłanym do okna bezpośrednio po jego wyświetleniu}
	{Nie}{Jest komunukatem wysyłanym do okna aplikacji informującym go o potrzebe przerysowania}
	
	\item \question{Czy dany przycisk został naciśnięty możemy sprawdzić poprzez:}
	{Tak}{Porównanie uchwytu do przycisku wewnątrz procedury obsługi komunikatów przy zdarzeniu \texttt{WM\_COMMAND}}
	{Nie}{Porównanie adresu kontrolki przycisku}
	{Tak}{Porównanie ID przypisanego do przycisku wewnątrz procedury obsługi komunikatów przy zdarzeniu \texttt{WM\_COMMAND}}
	{Nie}{Wykonanie procesury obsługi przerwania danego przycisku}
	
	\item \question{Wyświetlenie okna Message Box:}
	{Nie}{Powoduje utworzenie dla niego nowego procesu w systemie}
	{Tak}{Jest wywołaniem blokującym (blokuje wykonanie dalszej części kodu aż do zamknięcia Message Box'a)}
	{Nie}{Polega na obsłudze odpowiedniego komunikatu w pętli obsługi komunikatów.}
	{Tak}{Możemy uzyskać poprzez wywołanie kodu: MessageBox(NULL, L"Welcome to Win32 Application Development$\backslash$n", NULL, NULL);}
	
	\item \question{DefWindowProc}
	{Tak}{Jest domyślną funkcją obsługującą komunikaty napływające do okna aplikacji}
	{Nie}{Jest wskaźnikiem na funkcję obsługującą komunikaty napływające do okna aplikacji}
	{Nie}{Jest funkcją pozwalającą na zdefiniowanie głównego okna aplikacji}
	{Nie}{Jest strukturą pozwalająca na m.in. zdefiniowanie głównego okna aplikacji}

	\item \question{Jakie rodzaje komunikatów mogą docierać do okna?}
	{Tak}{zmiana rozmiaru okna}
	{Tak}{pojedyncze bądź podwójne kliknięcie myszą w obszarze okna}
	{Tak}{zmiana położenia okna}
	{Tak}{naciśnięcie klawisza}
	
	\item \question{WNDCLASS$\slash$WNDCLASSEX}
	{Nie}{Obsługuje kolejkę komunikatów napływających do okna aplikacji}
	{Nie}{Jest strukturą przechowującą wskaźniki do poszczególnych okien aplikacji}
	{Tak}{Jest strukturą pozwalającą zdefiniować np. koloty okna aplikacji}
	{Nie}{Jest odpowiednikiem funkcji main() w programach pisanych w WinAPI}
	
\end{enumerate}
	
	\newpage
% ======== LINUX ============================================= %
\part{Linux}
	% --- Usługi graficzne Xwindow --------- %
	\section{Usługi graficzne Xwindow}
	
	% --- Linux ACL ------------------------ %
	\newpage
\section{Linux ACL}

\begin{enumerate}
	
	\item \question{Efekt polecenia ls -l file.txt jest następujący: \\
		-rw-r-{}-{}-{}-{}- 1 me students 0 2010-02-20 23:10 file.txt \\ \\
		W następnym kroku powyższemu plikowi nadano pewne uprawnienia ACL, a następnie wykonano polecenie getfacl file.txt uzyskując następujący wynik: \\ \\
		$\sharp$file: file.txt \\
		$\sharp$owner: me \\
		$\sharp$group: students \\
		user::rw- \\
		user:friend:r-{}- \\
		group::r-{}- \\
		group: class:rw- \\
		mask::rw- \\
		other::-{}-{}- \\
		\\ Zaznacz poprawne polecenia, które mogłyby zostać wykonane w celu uzyskania powyższych uprawnieć ACL:}
	{Tak}{setfacl -m u:friend:4, g:class:6 file.txt}
	{Tak}{setfacl -m u:friend:r, g:class:rw file.txt}
	{Nie}{setfacl -m u:r:friend, g:rw:class file.txt}
	{Nie}{setfacl -x u:friend:4, g:class6 file.txt}
	
	\item \question{Efekt polecenia ls -l test jest następujący: \\
		drw-r-{}-{}-{}-{}- 1 so1 students 0 2011-06-10 23:10 test \\ \\
		W następnym kroku powyższemu plikowi nadano pewne uprawnienia ACL, a następnie wykonano polecenie getfacl test uzyskując następujący wynik: \\ \\
		$\sharp$file: test \\
		$\sharp$owner: so1 \\
		$\sharp$group: students \\
		user::rwx \\
		group::r-x \\
		other::r-x \\
		default:user::rwx \\
		default:group::r-x \\
		default:grup:teachers:rwx \\
		default:mask::rwx \\
		default:other::r-x \\
		\\ Zaznacz poprawne polecenia, które mogłyby zostać wykonane w celu uzyskania powyższych uprawnieć ACL:}
	{Tak}{setfacl -d -m g:teacher:rwx test}
	{Nie}{brak poprawnej odpowiedzi}
	{Nie}{setacl -m g:teacher:rwx test}
	{Nie}{nie istnieje żadne polecenie, które pozwalałoby uzyskać podany wynik}
	
	\newpage
	
	\item \question{Którym poleceniem można zmienić ustawienia pliku file, tak aby użytkownik user1 miał pełne uprawnienia, a grupa group1 mogła czytać i modyfikować, ale nie mogła go wykonać jako skryptu?}
	{Tak}{setfacl -m u:user1:7, g:group1:6 file}
	{Nie}{setfacl -m u:user1:r-x, g:group1:rw- file}
	{Nie}{setfacl -m u:user1:6, g:group1:7 file}
	{Tak}{setfacl -m u:user1:rwx, g:group1:rw- file}
	
	\item \question{Polecenie getfacl:}
	{Tak}{zwraca informacje na temat aktualnych uprawnień zdefiniowanych na liście ACL}
	{Nie}{usuwa uprawnienia zdefiniowane na liście ACL}
	{Tak}{zwraca informację na temat właściciela pliku}
	{Tak}{Pozwala wyświetlić informacje na temat uprawnień zdefiniowanych w ACL dla kilku plików na raz}
	
	\item \question{Zaznacz odpowiadające sobie mapowanie typów ACL na standardowe Linuxowe klasy użytkowników:}
	{Nie}{named user - owner}
	{Tak}{owner - owner}
	{Tak}{mask - group}
	{Nie}{owning group - group}
	
	\item \question{Polecenie, w wyniku którego każdy nowoutworzony PLIK będzie miał uprawnienia -rwxr-x-{}-{}- to:}
	{Nie}{umask 027}
	{Nie}{umask 750}
	{Nie}{umask 750}
	{Tak}{brak poprawnej odpowiedzi}
	
	\item \question{Polecenie setfacl -m u:user1:6, g:group1:7 file.txt:}
	{Nie}{Ustawi prawa do pliku "file.txt" wszystkich użytkowników jako rwx.}
	{Nie}{Umożliwi użytkownikowi o nazwie "user1" wykonanie pliku "file.txt".}
	{Tak}{Ustawi prawa do pliku "file.txt" użytkownika o nazwie "user1" jako rw-, a grupy o nazwie "group1" jako rwx.}
	{Nie}{Ustawi prawa do pliku "file.txt" użytkownika o nazwie "user1" jako r-{}-m a grupy o nazwie "group1" jako -{}-{}-.}
	
	\item \question{W systemie Linux Debian użytkownik wykonał sekwencję poleceń: \\
		umask 075; touch test; ls -l |grep test; \\
		Zaznacz poprawny wynik dla podanej sekwencji poleceń:}
	{Nie}{-{}-{}-rwxr-x 1 labso labso 0 2010-06-11 16:30 test}
	{Tak}{-rw-{}-{}-{}-w- labso labso 0 2010-06-11 16:30 test}
	{Nie}{-rwx-{}-{}-{}-wx 1 labso labso 0 2010-06-11 16:30 test}
	{Nie}{-rw-rw-r-{}- 1 labso labso 0 2010-06-11 16:30 test}
	
	\newpage
	
	\item \question{Wskaż poprawną odpowiedź dotyczącą instalacji ACL na komputerze z systemem ubuntu/debian:}
	{Nie}{ACL nie znajduje się oficjalnie w repozytorium. Należy pobrać źródła z internetu oraz samodzielnie przeprowadzić kompilację oraz konfigurację.}
	{Nie}{Nie jest wymagana instalacja ACL. Systemy te zawierają preinstalowane paczki związane z ACL.}
	{Nie}{Należy zainstalować acl komendą sudo apt-get install acl. Instalator automatycznie skonfiguruje system do pracy z ACL.}
	{Tak}{Należy zainstalować acl komendą sudo apt-get install acl, a następnie manualnie przeprowadzić konfigurację systemów plików w pliku /etc/fstav podłączając ACL.}
	
	\item \question{Uprawnienia dla nowo tworzonych plików przy masce 066 wyglądają następująco:}
	{Nie}{-rwxrwxrwx}
	{Nie}{-rw-rw-r-{}-}
	{Nie}{-{}-{}-rw-rw-}
	{Tak}{-rw-{}-{}-{}-{}-{}-}
	
	\item \question{W stosunku do chmod, lista ACL rozszerzyła możliwości przyznawania praw o:}
	{Tak}{Określenie praw do pliku dla dowolnej grupy.}
	{Tak}{Określenie praw do pliku dla dowolnego użytkownika.}
	{Nie}{Określenie praw do pliku dla innych - other.}
	{Nie}{Określenie praw do pliku dla właściciela - owner.}
	
	\item \question{W systemie Linux z działającym systemem ACL wydano polecenie getfacl mySong.bin. Otrzymano następujący wynik: \\
		 $\sharp$file: mySong.bin \\
		 $\sharp$owner: jan \\
		 $\sharp$group: homegroup \\
		 user::rw- \\
		 user:maria:r-{}- \\
		 group::r-{}- \\
		 group:dzieci:rw- \\
		 mask::rwx \\
		 other::-{}-{}-
		\\ W tym przypadku: }
	{Tak}{użytkownik z grupy dzieci może odczytywać plik mySong.bin}
	{Tak}{użytkownik maria może odczytywać plik mySong.bin}
	{Nie}{użytkownik maria może modyfikować plik mySong.bin}
	{Tak}{uzytkowik z grupy dzieci może modyfikować plik mySong.bin}
	
	\newpage
	
	\item \question{Zaznacz poprawne odpowiedzi dotyczące maski oraz wyznaczania uprawnień dla wpisów ACL powiązanych z klasą grupy:}
	{Tak}{Maska definiuje maksymalne efektywne uprawnienia dla wszystkich wpisów ACL powiązanych z klasą grupy}
	{Nie}{Uprawnienia efektywne powstają przez zsumowanie uprawnień maski z uprawnieniami odpowiedniej klasy ACL}
	{Nie}{Maska definiuje minimalne efektywne uprawnienia dla wszystkich wpisów ACL powiązanych z klasą grupy}
	{Tak}{Uprawnienia efektywne powstają przez przecięcie uprawnień maski z uprawnieniami odpowiedniej klasy ACL}

	\item \question{Wskaż poprawne stwierdzenia dotyczące Linux ACL}
	{Tak}{Uprawnienie typu named-group można zamaskować}
	{Tak}{Maska w Linux ACL określa maksymalne uprawnienia}
	{Nie}{Uprawnienie wpisu ACL other można zamaskować}
	{Tak}{Uprawnienie typu named-user można zamaskować}
	
	\item \question{Aby korzystać w systemie Linux z Acces Control List (ACL) należy:}
	{Nie}{ACL jest domyślnie włączony zaraz po instalacji dystrybucji systemu Linux.}
	{Tak}{Dodać obsługę ACL do wszytskich systemów plików w pliku /etc/fstab.}
	{Nie}{Żadna odpowiedź nie jest poprawna.}
	{Tak}{Zainstalować pakiet acl.}
	
	\item \question{Efekt polecenia ls -l test.txt jest następujący: \\
		-rw-r-{}-{}-{}-{}- 1 so1 students 0 2011-06-10 23:10 test \\ \\
		W następnym kroku powyższemu plikowi nadano pewne uprawnienia ACL, a następnie wykonano polecenie getfacl test.txt uzyskując następujący wynik: \\ \\
		$\sharp$file: test \\
		$\sharp$owner: so1 \\
		$\sharp$group: students \\
		user::rwx \\
		group::r-x \\
		other::r-x \\
		default:user::rwx \\
		default:group::r-x \\
		default:group:teachers:rwx \\
		default:mask::rwx \\
		default:other::r-x \\
		default:other::-{}-{}-
		\\ Zaznacz poprawne polecenia, które mogłyby zostać wykonane w celu uzyskania powyższych uprawnieć ACL:}
	{Nie}{brak poprawnej odpowiedzi}
	{Nie}{setfacl -d -m g:teachers:rwx test}
	{Nie}{setfacl -m g:teachers:rwx test}
	{Tak}{Nie istnieje żadne polecenie, które pozwalałoby uzyskać podany wynik}
	
	\newpage
	
	\item \question{W jaki sposób można sprawdzić, czy dany plik ma zdefiniowane dodatkowe uprawnienia ACL?}
	{Tak}{Poprzez użycie polecenia getfacl}
	{Nie}{Poprzez użycie polecenia filefrag}
	{Nie}{Korzystając z polecenia ps z argumentem -aux}
	{Tak}{Używając polecenia ls}
	
	\item \question{Polecenie, wyniku którego każdy nowoutworzony KATALOG w systemie Debian będzie miał uprawnienia 644 to:}
	{Nie}{umask 644}
	{Nie}{brak poprawnej odpowiedzi}
	{Tak}{umask 133}
	{Nie}{umask 022}
	
	\item \question{Efekt polecenia ls -l file.txt jest następujący: \\
		-rw-r-{}-{}-{}-{}- 1 so1 students 0 2010-02-20 23:10 test.txt \\ \\
		W następnym kroku powyższemu plikowi nadano pewne uprawnienia ACL, a następnie wykonano polecenie getfacl test.txt uzyskując następujący wynik: \\ \\
		$\sharp$file: test.txt \\
		$\sharp$owner: so1 \\
		$\sharp$group: students \\
		user::rw- \\
		user:so2:rw- \\
		group::r-{}- \\
		group: teachers:rwx \\
		mask::rwx \\
		other::-{}-{}- \\
		\\ Zaznacz poprawne polecenia, które mogłyby zostać wykonane w celu uzyskania powyższych uprawnieć ACL:}
	{Tak}{setfacl -m u:so2:rw, g:teachers:rwx test.txt}
	{Tak}{setfacl -m u:so2:6, g:teachers:7 test.txt}
	{Nie}{setfacl -x u:so2:rw, g:teachers:rwx test.txt}
	{Nie}{setfacl -m user:rw:so2, group:rwx:teachers test.txt}
		
\end{enumerate}
	
	% --- Linux RAID ----------------------- %
	%\item \question{}%
%{Tak}{}%
%{Nie}{}%
%{}{}%
%{}{}

% !TeX spellcheck = pl_PL
% *****************************************************
% Możliwe, że coś się wymieszało z Windowsem,
% do sprawdzenia bliżej laborek
% *****************************************************
\newpage
\section{Linux RAID}
	\begin{enumerate}
		\item \question{Macierz typu raid 5 złożona z 3 dysków o jednakowej pojemności i parametrach:}%
		{Nie}{ma pojemność 2 dysków i nie jest odporna na awarię ani jednego dysku}%
		{Tak}{oferuje spowolniony odczyt w przypadku awarii 1 dysku}%
		{Nie}{ma pojemność 1 dysku i jest odporna na awarię maksymalnie 2 dysków}%
		{Tak}{ma pojemność 2 dysków i jest odporna na awarię maksymalnie 1 dysku}
		\item \question{W systemie Ubuntu, zakładając, że pliki blokowe /dev/sdb1 i /dev/sdb2 reprezentują partycje o rozmiarze 50MB, bezpośrednio po utworzeniu woluminu /dev/md0 poleceniem:\\
		mdadm $ -- $create $ -- $verbose /dev/md0 $ -- $level=linear $ -- $raid-devices=2\\/dev/sdb1/dev/sdb2:}%
		{Tak}{wolumin /dev/md0 będzie miał wielkość 100MB}%
		{Nie}{wolumin /dev/md0 będzie miał wielkość 50MB}%
		{Nie}{wolumin /dev/md0 będzie można zamontować poleceniem mount /dev/md0 /mnt}%
		{Tak}{uszkodzenie dokładnie jednego spośród urządzeń /dev/sdb1 oraz /dev/sdb2 może spowodować utratę danych w woluminie /dev/md0}
		\item \question{Zaznacz prawdziwe stwierdzenia:}%
		{Tak}{Sprzętowy RAID oferuje większą wydajność poprzez zmniejszenie obciążenia CPU, gdyż przeliczaniem sum kontrolnych zajmuje się wówczas dedykowany kontroler.}%
		{Nie}{RAID sprzętowy jest niekompatybilny z dużą liczbą systemów operacyjnych, ze względu na zachowanie odróżniające taki RAID od pojedynczego dysku twardego.}%
		{Tak}{RAID software'owy oferuje możliwość łączenia różnych interfejsów takich jak ATA, SCSI, SATA, USB w obrębie jednej macierzy.}%
		{Nie}{Dla takich samych dysków RAID 6 oferuje większą szybkość zapisu niż RAID 0.}
		\item \question{RAID5 może składad się z następującej ilości dysków:}%
		{Nie}{2}%
		{Tak}{3}%
		{Tak}{4}%
		{Tak}{5}
		\item \question{RAID inaczej zwanym lustrzanym (mirroringiem) to:}%
		{Tak}{RAID1}%
		{Nie}{RAID2}%
		{Nie}{RAID3}%
		{Nie}{RAID5}
		\item \question{Jakie polecenie pozwoli na rozpoczecie procedury tworzenia partycji:}%
		{Tak}{fdisk /dev/hda}%
		{Nie}{mkdir /dev/sda}%
		{Tak}{fdisk /dev/sdb}%
		{Nie}{mdadd /dev/sdb}
		\item \question{Jaka ilość dysków jest wystarczająca, aby zastosować RAID:}%
		{Nie}{1}%
		{Nie}{2}%
		{Tak}{3}%
		{Tak}{4}
		
		\newpage
		\item \question{Mając do dyspozycji 3 identyczne dyski twarde można stworzyć macierz RAID w konfiguracji:}%
		{Tak}{RAID 0}%
		{Tak}{RAID 5}%
		{Nie}{RAID 6}%
		{Nie}{RAID 10}
		\item \question{Trzy dyski zostały połączone w macierz RAID 0.}%
		{Nie}{Łączna przestrzeń dyskowa jest równa sumie przestrzeni, każdego z dysków}%
		{Tak}{Łączna przestrzeń dyskowa jest równa potrojonej przestrzeni dyskowej najmniejszego dysku}%
		{Tak}{Szybkość jest równa potrojonej szybkości najwolniejszego z dysków}%
		{Nie}{Szybkość jest równa szybkości najwolniejszego z dysków}
		\item \question{Zaznacz cele zastosowania macierzy RAID:}%
		{Tak}{Zwiększenie odporności na awarie}%
		{Tak}{Zwiększenie wydajności transmisji danych}%
		{Tak}{Powiększenie przestrzeni dyskowej, dostępnej jako jedna całość}%
		{Nie}{Dwukrotne zwiększenie całkowitej przestrzeni dyskowej}
		\item \question{Administrator podłączył do komputera dwa dyski twarde o pojemności 200GB każdy i połączył je w macierz RAID 1. Do komputera nie zostały podłączone żadne inne dyski. Które z poniższych twierdzeń są prawidłowe?}%
		{Tak}{Całkowita pojemność partycji dostępnych w systemie nie przekracza 200GB.}%
		{Nie}{Rozwiązanie takie zapewnia o wiele większą prędkość odczytu i zapisu danych niż macierz RAID 0.}%
		{Tak}{Rozwiązanie takie zapewnia o wiele większe bezpieczeństwo danych niż macierz RAID 0.}%
		{Nie}{W przypadku awarii jednego dysku użytkownik straci wszystkie swoje dane}
		\item \question{Zaznacz zdania prawdziwe dotyczące sprzętowej macierzy RAID:}%
		{Tak}{Macierz jest zupełnie przezroczysta, przez co z punktu widzenia Systemu Operacyjnego zachowuje się ona jak każdy inny dysk twardy}%
		{Nie}{mniejsza wydajność poprzez zwiększenie obciążenia CPU}%
		{Tak}{Minimalna liczba dysków potrzebna do stworzenia macierzy to 2}%
		{Nie}{Sprzętowa macierz RAID zawsze umożliwia przywrócenie danych w razie awarii jednego z dysków}
		\item \question{Zaznacz zdania prawdziwe dotyczące programowej macierzy RAID:}%
		{Nie}{Macierz jest zupełnie przezroczysta, przez co z punktu widzenia Systemu Operacyjnego zachowuje się ona jak każdy inny dysk twardy}%
		{Tak}{mniejsza wydajność poprzez zwiększenie obciążenia CPU}%
		{Tak}{Minimalna liczba dysków potrzebna do stworzenia macierzy to 2}%
		{Nie}{Programowa macierz RAID zawsze umożliwia przywrócenie danych w razie awarii jednego z dysków}
		
		\newpage
		\item \question{System Linux pozwala na:}%
		{Tak}{Tworzenie programowych macierzy RAID.}%
		{Tak}{Tworzenie wolumenów liniowych.}%
		{Nie}{Tworzenie partycji za pomocą polecenia "create"}%
		{Tak}{Tworzenie macierzy RAID 5.}
		\item \question{Woluminy liniowe w katalogu dev oznaczone są jako:}%
		{Tak}{md0,md1,...}%
		{Nie}{ma0,ma1,...,mb0,mb1,...}%
		{Nie}{raid0,raid1,...}%
		{Nie}{rda0,rda1,...,rdb0,rdb1,...}
		\item \question{Za pomocą polecenia mdadm można:}%
		{Tak}{utworzyć wolumin liniowy}%
		{Nie}{Sformatować partycję}%
		{Tak}{Sprawdzić konfigurację macierzy}%
		{Tak}{Zasymulować awarię woluminu}
		\item \question{Która z aplikacji umożliwia stworzenie partycji na twardym dysku?}%
		{Nie}{/etc/fstab}%
		{Tak}{/sbin/fdisk}%
		{Tak}{/sbin/cfdisk}%
		{Nie}{/etc/mtab}
		\item \question{Wskaż poprawne zdania dotyczące RAID.}%
		{Nie}{Polecenie „mdadm -C -v /dev/md0 --level=0 -n 2 /dev/sda1 /dev/sdb1” służy do stworzenia wolumenu liniowego na partycjach sda1 i sdb1.}%
		{Tak}{Polecenie „mdadm -C -v /dev/md0 –level=1 -n 2 /dev/sda1 /dev/sdb1” służy do stworzenia mirroru.}%
		{Tak}{Polecenie „mkfs -t ext3 /dev/md0” służy do sformatowania urządzenia.}%
		{Nie}{Wolumenu liniowego /dev/md0 nie można dodać do pliku /etc/fstab, aby była montowana przy starcie systemu operacyjnego.}
		\item \question{Które z wymienionych rodzajów macierzy RAID zapewniają mirroring:}%
		{Nie}{RAID 0}%
		{Tak}{RAID 1}%
		{Tak}{RAID 5}%
		{Tak}{RAID 10}
		\item \question{Które z wymienionych poleceń umożliwia zarządzanie macierzami RAID w systemie GNU/Linux:}%
		{Nie}{hdparm}%
		{Tak}{mdadm}%
		{Nie}{fdisk}%
		{Nie}{parted}
		
		\newpage
		\item \question{Celem wyłączenia automatycznego montowania urządzenia cdrom w systemie Linux należy:}%
		{Tak}{Odpowiednio zmodyfikować plik '/etc/fstab'.}%
		{Nie}{Wykonać polecenie 'nmount -n cdrom'.}%
		{Nie}{Wykonać polecenie 'nmount cdrom'.}%
		{Nie}{Odpowiednio zmodyfikować plik '/etc/amount'.}
		\item \question{Polecenie 'fdisk' w systemie Linux można wykorzystać do:}%
		{Tak}{tworzenia partycji.}%
		{Tak}{wypisania informacji o dysku.}%
		{Nie}{montowania dysku.}%
		{Nie}{tworzenia kopii zapasowej danych.}
		\item \question{Wskaż poprawne odpowiedzi dotyczące RAID5:}%
		{Tak}{Umożliwia odzyskanie danych w razie awarii jednego z dysków}%
		{Nie}{Składa się z minimum 2 dysków}%
		{Nie}{Odzyskiwanie danych w razie awarii odbywa się przy wykorzystaniu danych i kodów korekcyjnych zapisanych na jednym, specjalnie do tego przeznaczonym dysku}%
		{Tak}{W przypadku awarii dysku dostęp do danych jest spowolniony}
		\item \question{Wskaż poprawne odpowiedzi dotyczące mirroring-u:}%
		{Tak}{Polega na zapisywaniu tych samych danych na dwóch lub więcej dyskach jednocześnie}%
		{Nie}{W przypadku awarii co najmniej połowy z dysków nie ma możliwości odzyskania wszystkich danych}%
		{Tak}{Dostępna przestrzeń ma rozmiar najmniejszego nośnika}%
		{Tak}{Czas równoległego zapisu jest równy czasowi zapisu na najwolniejszym dysku}
		\item \question{Wskaż poprawne zdania dotyczące RAID5 w systemie Linux:}%
		{Nie}{Do utworzenia RAID5 potrzebne są co najmniej dwie partycje.}%
		{Nie}{Do utworzenia RAID5 można użyć maksymalnie trzech partycji.}%
		{Nie}{Do odtworzenia danych z uszkodzonej partycji zawsze wykorzystywana jest jedna, specjalnie do tego przygotowanej partycja.}%
		{Tak}{RAID5 jest całkowicie odporny na uszkodzenie jednej partycji (dane można w pełnie odtworzyć).}
		\item \question{Wskaż poprawne zdania dotyczące RAID1 (mirror) w systemie Linux.}%
		{Tak}{Całkowita pojemność partycji połączonych w RAID1 jest taka jak pojemność najmniejszej z tych partycji.}%
		{Tak}{Do utworzenia RAID1 można wykorzystać trzy partycje.}%
		{Nie}{Zastosowanie RAID1 pozwala na zwiększenie szybkości zapisu i odczytu danych.}%
		{Tak}{RAID1 jest całkowicie odporny na uszkodzenie jednej partycji (dane można w pełni odtworzyć).}
		
		
		
		
		
		
		
		
		
		
		
		
		
		
		
		
		
		
		
		
		
		
		
		
		
		
		
		
		
	\end{enumerate}
	
	% --- Linux LAMP ----------------------- %
	%\item \question{}%
%{Tak}{}%
%{Nie}{}%
%{}{}%
%{}{}

% !TeX spellcheck = pl_PL
\newpage
\section{Linux LAMP}
\begin{enumerate}
	\item \questionVIII{%
			question=Zaznacz wszystkie poprawne stwierdzenia dotyczące rozwiązania LAMP: %
		}{%
			isTrue1=Nie, %
			answer1=Konfiguracja baz danych może odbywać się wyłącznie poprzez narzędzie phpMyAdmin., %
			isTrue2=Nie, %
			answer2=MySQL pozwala na wykonywanie kodu zapisanego w języku PHP na stronie wwww., %
			isTrue3=Tak, %
			answer3=Funkcją MySQL jest zarządzanie bazą danych., %
			isTrue4=Tak, %
			answer4=Podstawową funkcją serwera Apache jest przesyłanie do klienta treści plików znajdujących się na dysku przy wykorzystaniu protokołu HTTP., %
			isTrue5=Nie, %
			answer5=Kod PHP wewnątrz pliku z rozszerzeniem .html może znajdować się pomiędzy znacznikiem $ < $php$ > $ oraz znacznikiem $ < $/php$ > $., %
			isTrue6=Tak, %
			answer6=Kod PHP wewnątrz pliku z rozszerzeniem .php może znajdować się pomiędzy znacznikiem $ < $? oraz znacznikiem ?$ > $., %
			isTrue7=Tak, %
			answer7=Pliki konfiguracyjne serwera Apache znajdują się w katalogu /etc/apache2/, %
			isTrue8=Nie, %
			answer8=phpMyAdmin jest narzędziem do konfiguracji w trybie tekstowym. %
		}
\end{enumerate}






	
	% --- Wielosystemowość ----------------- %
	\newpage
\section{Wielosystemowość}
\begin{enumerate}

	\item \question{Po zmianie w plikach konfiguracyjnych programu GRUB:}
	{Tak}{zmiany NIE SĄ automatycznie wprowadzone po zmianie zawartości plików}
	{Tak}{nalezy wydać polecenie update-grub jako root, aby konfiguracja nowa konfiguracja została wprowadzona}
	{Nie}{zmiany od razu nie są wprowadzone, zaraz po zmianie pliku}
	{Nie}{plików konfiguracyjnych GRUBa nie wolno edytować (jest to robione automatycznie przez system)}
	
	\item \question{Wksaż poprawne zdanie na temat dysku /dev/sdd3}
	{Nie}{Jest to czwarta partycja czwartego dysku SATA}
	{Nie}{Jest to czwarta partycja trzeciego dysku SATA}
	{Nie}{Oznaczenie nie jest poprawne}
	{Tak}{Jest to trzecia partycja czwartego dysku SATA}
	
	\item \question{Czym charakteryzuje się plik konfiguracyjny "grub.cfg" menedżera GRUB 2, znajdujący się standardowo w katalogu "/boot/grub"?}
	{Nie}{Jest to jedyny plik konfiguracji GRUB 2, którego własnoręczna edycja nie jest odradzana}
	{Tak}{Nie powinien być bezpośrednio edytowany przez użytkownika.}
	{Tak}{Może zostać nadpisany w wyniku polecenia "update-grub".}
	{Tak}{Zawiera wpisy dotyczące uruchamianych systemów operacyjnych.}
	
	\item \question{Polecenie mount -a}
	{Nie}{montuje wszystkie systemy plików wylistowane w pliku /etc/fstab}
	{Tak}{montuje systemy plików wylistowane w pliku /etc/fstab, które nie korzystają z opcji noauto}
	{Nie}{może być wykonane przez dowolnego użytkownika}
	{Tak}{zarezerwowane jest tylko dla roota}
	
	\item \question{Wskaż, które z poniższych twierdzeń odnoszących się do pliku konfiguracyjnego "/etc/fstab" są poprawne.}
	{Tak}{Definiując poszczególne systemy plików możemy posłużyć się zarówno unikalnym identyfikatorem dysku, jak i nazwą urządzenia.}
	{Nie}{Edytując plik użytkownik może wskazać jako miejsce montowania nieistniejący katalog, w trakcie uruchomienia systemu, katalog taki zostanie utworzony.}
	{Tak}{Plik ten zawiera informację na temat wszystkich systemów plików, które powinny być montowane w trakcie uruchamiania systemu.}
	{Tak}{Do edycji pliku wymagane są uprawnienia administratora.}
	
	\item \question{Używając bootloader'a GRUB2:}
	{Tak}{hd1 oznacza drugi dysk w systemie (/dev/sdb)}
	{Nie}{hd1 oznacza pierwszy dysk w systemie (/dev/sda)}
	{Tak}{setroot(hd0, 1) odwoła się do pierwszej partycji pierwszego dysku (dev/sda1)}
	{Nie}{setroot(hd0, 1) odwoła się do drugiej partycji pierwszego dysku (dev/sda2)}
	
	\item \question{Co spowoduje dodanie następującego wpisu do pliku /etc/grub.d/4-\_custom \\
		menuentry "Windows" \{ \\
		ser root='(hd0,1)' \\
		chainloader + 1 \\
		\} }
	{Nie}{Podczas startu bootloadera będziemy mogli wybrać system o nazwie "Windows" i będzie one pierwszy na liście dostępnych systemów.}
	{Nie}{Jest to niepoprawny wpis.}
	{Tak}{Podczas startu bootloadera będziemy mogli wybrać system o nazwie "Windows", znajdujący się na dysku "hd0".}
	{Tak}{W celu załadowania systemu Windows sterowanie zostanie przekazane do pierwszego sektora z podanej partycji (zostanie uruchomiony kod, który się tam znajduje).}

	\item \question{Program Grub pozwala na:}
	{Nie}{Rekompilację jądra Linux}
	{Tak}{Automatyczne uruchomienie wybranego systemu z pominięciem wyświetlania ekranu wyboru.}
	{Nie}{Zarządzanie dyskami i ich partycjonowanie}
	{Tak}{Wybór systemu operacyjnego, który będzie uruchomiony.}

	\item \question{Parametr w opcjach montowania pliku /etc/fstab oznacza, że:}
	{Nie}{możliwy jest zapis i odczyt na danym systemie plików}
	{Tak}{system plików jest zamontowany w trybie tylko do odczytu}
	{Nie}{urządzenie może być montowane przez użytkownika}
	{Nie}{system plików może być montowany przez każdego użytkownika}
	
	\item \question{W jaki sposób dodajemy informacje o innych systemach opracyjnych do GRUB2}
	{Tak}{Do pliku /etc/grub.d/40\_custom dodajemy wpis o systemie, następnie uruchamiamy polecenie sidu update-grub2}
	{Nie}{Należy wykonanać polecenie grub2-add-new-os z prawami użytkownika}
	{Tak}{Można nadać prawa wykonywania skryptowi: /etc/grub.d/30\_od-prober. Grub2 podczas aktualizacji wyszuka dostępne systemy operacyjne na dyskach twardych}
	{Nie}{GRUB2 sam wykryje wszystkie systemy operacyjne bez konfiguracji}
	
	\item \question{Plik /boot/grub.cfg dla Grand United Bootloader w wersji 2:}
	{Tak}{posiada definicje wszystkich systemów uruchamianych przez niego}
	{Tak}{w przypadku edycji za każdym razem musi być zaktualizowany za pomocą polecenia update-grub}
	{Tak}{Tworzony jest automatycznie na podstawie skryptów znajdujących się w ktalogu /etc/grub.d/}
	{Nie}{Tworzony jest automatycznie na podstawie konfiguracji zdefiniowanej w pliku /etc/grub/default}
	
	\newpage

	\item \question{Wskaż, które z poniższych twierdzeń odnoszących się do bootmanagera GRUB2 są poprawne.}
	{Nie}{Aby zablokować możliwość wykonywania się danego skryptu podczas aktualizacji GRUB'a wystarczy odebrać mu uprawnienia do odczytu.}
	{Tak}{Lista zdefiniowanych, uruchamianych przez GRUB2 systemów operacyjnych zdefiniowana jest w pliku "/boot/grub/grub.cfg".}
	{Tak}{Wywołanie polecenia "update-grub" powoduje uruchomienie skryptów umieszczonych w katalogu "/etc/grub.d"}
	{Nie}{Po wywołaniu polecenia "update-grub" skrypt "30\_os-prober" zostanie uruchomiony przed skryptem "10\_linux".}
	
	\item \question{Jakim poleceniem tworzony (bądź aktualizowany) jest plik konfiguracyjny /boot/grub.grub.cfg?}
	{Nie}{grub-config}
	{Nie}{grub-install}
	{Nie}{grub-refresh}
	{Tak}{update-grub}
	
	\item \question{Na jednym fizycznym komputerze, na osobnych partycjach są zainstalowane systemu buntu Linux i Windows 7. Przy obecnej konfiguracji użytkownik mam możliwość (przy użyciu bootmanagera GRUB 2) uruchomienia TYLKO systemu Ubuntu. W jaki sposób można zapeewić użytkownikowi możliwość wyboru systemu operacyjnego przy uruchamianu komputera?}
	{Tak}{Należy utworzyć własny plik z odpowiednim wpisem systemu oraz prawami uruchamiania w /etc/grub.d/, a następnie zaktualizować pliki konfiguracyjne GRUB'a}
	{Tak}{Dodać odpowiedni wpis w pliku /boot/grub/grub.cfg}
	{Nie}{Należy włożyć dysk instalacyjny Windowsa i z linii poleceń, za pomocą komendy bootrec /fixmbr zainstalować w MBR bootloader dla systemu Windows}
	{Tak}{Ustawić prawa uruchamiania dla skrypty /etc/grub,d/30\_os-prober oraz uruchomić update-grub}

	\item \question{Zaznacz, które z podanych plików w systemach z rodziny Linux zawierają informacje o systemach, które mają zostać automatycznie zamontowane przy uruchomieniu systemu operacyjnego.}
	{Nie}{/boot/grub/grub.cfg}
	{Nie}{/etc/default/grub}
	{Nie}{/etc/mtab}
	{Tak}{/etc/fstab}
	
	\item \question{Wskaż wszystkie poprawne odpowiedzi dotyczące bootmanagera GRUB2}
	{Tak}{Skrypty konfiguracyjne znajdujące się w katalogu /etc/grub.d/ uruchamiane są w momencie wywołania grub-update}
	{Nie}{Nie wymaga aktualizowania pliku /etc/boot/grub.cfg po wprowadzeniu zmian do pliku konfiguracyjnego /etc/default/grub - zawartość tego pliku odczytywana jest na bieżąco w momencie uruchamiania systemu.}
	{Tak}{Jest domyślnym managerem bootowania systemu Linux Ubuntu od dystybucji 9.10}
	{Tak}{Plik /boot/grub/grub.cfg jest jednym z najistotniejszych plików konfiguracyjnych managera GRUB2}
	
	\newpage
	
	\item \question{Program fdisk}
	{Tak}{Pozwala na sformatowanie wybranej partycji}
	{Tak}{Wywołany z parametrem -i wyświetla tablice partycji dla podanych urządzeń}
	{Nie}{Pozwala na obsługę tablicy partycji systemu linux}
	{Nie}{Zmiany wprowadzone za pomocą tego programu automatycznie modyfikują zawartość plików /etx/fstab i /etc/mtab}

	\item \question{Jeżeli nie chcemy, aby konfiguracja zdefiniowana w pewnym skrypcie konfiguracyjnym GRUBA znajdującym się w katalogu /etc/grub.d/ była uwzględniona po wykonaiu polecenia update-grub, należy:}
	{Nie}{Zabrać temu skryptowi uprawnienia zapisu}
	{Nie}{Wprowadzić odpowiednie zmiany w pliku /etc/default/grub}
	{Nie}{Zabrać temu skryptowi uprawnienia odczytu}
	{Tak}{Zabrać temu skryptowi uprawnienia wykonywalności}
	
	\item \question{Plik /etc/fstab zawiera informacje o:}
	{Tak}{systemach plików montowanych podczas uruchomienia systemu}
	{Nie}{aktualnie zamontowanych systemach plików}
	{Nie}{tablicach partycji na aktualnie podłączonych dyskach}
	{Nie}{mapowaniu identyfikatorów UUID na oznaczenia linuksowe (sda, sdb, itd.)}
	
	\item \question{Wskaż prawdziwe zdania:}
	{Tak}{Plik /boot.grub/grub.cfg jest generowany automatycznie na podstawie skryptów z katalogu /etc/grub.d/}
	{Tak}{Pod Windowsem możliwe jest odczytywanie partycji ext2/ext3 za pomocą dodatkowego oprogramowania}
	{Nie}{Pod Linuksem jest możliwość obsługi partycji NTFS, ale jedynie w trybie do odczytu}
	{Nie}{GRUB jest w stanie uruchamiać jedynie Linuksa i Windowsa}
	
	\item \question{Domyślnie skrypt /etc/grub.d/30\_os-prober}
	{Nie}{ustawia tło, kolory tekstu, motyw graficzny}
	{Nie}{lokalizuje jądra hurd}
	{Nie}{lokalizuje jądro Linuksa}
	{Tak}{wyszukuje w każdej partycji systemów operacyjnych i integruje je w startowym menu}

	\item \question{Plik /etc/mtab przechowuje informacje o:}
	{Nie}{Systemach plików montowanych przy starcie systemu}
	{Tak}{Aktualnie zamontowanych systemach plików}
	{Nie}{Systemach plików oczekujących na zamontowanie w systemie}
	{Nie}{Systemach plików, które z jakiś powodów nie mogły zostać zamontowane i pojawić się tym samym pliku /etc/fstab}
	
	\newpage
	
	\item \question{Dodanie systemu operacyjnego do menu GRUB'a może nastąpić w wyniku}
	{Tak}{wykonania standardowego skryptu 30\_os-prober, a następnie wykonania polecenia update-grub}
	{Tak}{stworzenia własnego skryptu w katalogu /etc/grub.d/, a następnie wykonania polecenia update-grub}
	{Nie}{dodania odpowiedniego wpisu do pliku device.map, a następnie wykonania polecenia update-grub}
	{Tak}{dodania odpowiedniego wpisu do pliku 40\_custom, a następnie wykonania polecenia update-grub}
	
	\item \question{Wskaż wszystkie poprawne zdania odnoścnie pliku device.map}
	{Tak}{Ręczna zmiana pliku device.map wymaga aktualizacji konfiguracji GRUBa}
	{Tak}{Zawiera zmapowane nazwy urządzeń GRUBa na nazwy Linuxowe}
	{Nie}{Po każdym restracie systemu zapisywana jest do niego aktualna struktura dysków.}
	{Nie}{W wersji bootloadera GRUB2 plik ten nie istnieje}
	
	\item \question{Parametr ro w opcjoach montowania pliku etc/fstab oznacza, że:}
	{Nie}{możliwy jest zapis i odczyt na danym systemie plików}
	{Tak}{system plików jest zamontowany w trybie tylko do odczytu}
	{Nie}{urządzenie może być montowane przez użytkownika}
	{Nie}{system plików może być montowany przez każdego użytkownika}
	
	\item \question{Dodajemy własny wpis do menu GRUB2. Które z poniższych wartości parametru "setroot" bloku menuentry są poprawne?}
	{Nie}{setroot = (hda,1)}
	{Tak}{setroot = (hd0, msdos1)}
	{Nie}{setroot = (sda,1)}
	{Tak}{setroot = (hd0,1)}
	
	\item \question{Informacje na temat wszystkich systemów plików, które mają być montowane podczas uruchamiania systemu znajdują się w pliku:}
	{Nie}{/mnt}
	{Tak}{/etc/fstab}
	{Nie}{/etc/default/fstab}
	{Nie}{/etc/mtab}
	
	\item \question{Plik /boot/grub/grub.cfg zawiera:}
	{Nie}{tryb, w jakim ma się ładować system.}
	{Tak}{liste systemów operacyjnych, które można uruchomić za pomocą GRUBa}
	{Tak}{informację o tym, który sytem jest systemem domyślnym.}
	{Tak}{czas oczekiwania na wybór systemu przez użytkownika, po upływie którego uruchomi się domyślny system.}
	
	\newpage
	
	\item \question{Jakie informacje na temat zamontowanych systemów plików znajdują się w /etc/fstab?}
	{Nie}{Data zamontowania urządzenia.}
	{Tak}{Miejsce zamontowania systemu plików}
	{Tak}{Typ systemu plików.}
	{Nie}{Wielkość partycji.}

	\item \question{Zaznacz zdania poprawne dotyczące odwoływania się do systemów plików w systemie Linux.}
	{Tak}{/dev/fd0 - oznacza dyskietkę/}
	{Tak}{/dev/hdd2 - oznacza drugą partycję znajdującą się na dysku "slave" podpiętego do drugiego kontrolera IDE.}
	{Nie}{/dev/sda1 - oznacza pierwszą partycję pierwszego dysku SCSII lub drugą partycję na kontrolerze SATA1.}
	{Nie}{/dev/ssd1 - oznacza pierwszą partycję dysku stworzonego w oparciu o technologię SSD}
	
	\item \question{Plik /etc/fstab:}
	{Tak}{może być modyfikowany przez administartora systemu}
	{Nie}{zawiera informacje o aktualnie zalogowanych użytkownikach}
	{Tak}{Jest odczytywany w trakcie uruchamiania systemu operacyjnego}
	{Tak}{zawiera informacje o systemach plików, jakie mają być montowane w trakcie uruchamiania systemu.}
		
\end{enumerate}
	
	% --- Linux Kernel --------------------- %
	% !TeX spellcheck = pl_PL
\newpage
\section{Linux Kernel}

\begin{itemize}
	
	
	\item \questionVIII{%
		question={Zaznacz wszystkie poprawne odpowiedzi:}%
	}{%
		isTrue1=Nie, %
		answer1={Jądro Linuxa jest mikrojądrem (microkernel)}, %
		isTrue2=Nie, %
		answer2={Jądro Linuxa jest jądrem typu hybrydowego (hybrid)}, %
		isTrue3=Tak, %
		answer3={Jądro Linuxa jest jądrem typu monolitycznego (monolythic)}, %
		isTrue4=Nie, %
		answer4={Jądro Linuxa jest napisane w C++}, %
		isTrue5=Nie, %
		answer5={Jądro Linuxa wykorzystuje bibliotekę libc (dzięki temu można wykorzystywać np. funkcję printf()}, %
		isTrue6=Tak, %
		answer6={Jądro Linuxa jest napisane w C},
		isTrue7=Tak,
		answer7={Jądro Linuxa zarządza pamięcią operacyjną (przydziały/zwolnienia)}.
	}
	
	\item \questionVIII{%
		question={Zaznacz wszystkie poprawne odpowiedzi:}%
	}{%
		isTrue1=Nie, %
		answer1={Do sterowania parametrami pracy jądra można wykorzystać pliki znajdujące się w katalogu \textbf{/var}}, %
		isTrue2=Tak, %
		answer2={Do sterowania pracą jądra Linuxa można wykorzystać polecenie \textbf{sysctl}}, %
		isTrue3=Tak, %
		answer3={Do jądra systemu operacyjnego Linux można, w czasie jego pracy, dołączać różnorodną funkcjonalność (np. sterowniki urządzenia)}, %
		isTrue4=Nie, %
		answer4={Do załadowania modułu w jądrze można wykorzystać polecenia rmmod oraz modprobe -r}, %
		isTrue5=Tak, %
		answer5={Do sterowania parametrami pracy jądra można wykorzystać pliki znajdujące się w katalogu \textbf{/proc}}, %
		isTrue6=Nie, %
		answer6={Do sterowania pracą jądra Linuxa można wykorzystać polecenie \textbf{sysinfo}}, %
		isTrue7=Nie, %
		answer7={Do usunięcia modułu z jądra można wykorzystać polecenie insmod}, %
		isTrue8=Tak, %
		answer8={Do sprawdzenia jakie moduły załadowane są do jądra można wykorzystać polecenie lsmod}, %
		isTrue9=Tak, %
		answer9={Do załadowania modułu w jądrze można wykorzystać polecenie modprobe oraz insmod}, %
		isTrue10=Tak, %
		answer10={Katalog \textbf{/proc} zawiera pliki, pozwalające na zmianę sposobu przydzielania pamięci programom przez jądro Linux},%
		isTrue11=Nie, %
		answer11={Katalog \textbf{/var} zawiera pliki, pozwalające na zmianę sposobu przydzielania pamięci programom przez jądro systemu Linux}, %
		isTrue12=Tak, %
		answer12={Do usunięcia modułu z jądra można wykorzystać polecenia modprobe oraz mmod},
		isTrue13=Tak, %
		answer13={Katalogi /proc, /sys oraz polecenie sysctl pozwalają na kontrolę pracy systemu},
		isTrue14=Tak,
		answer14={Z jądra systemu operacyjnego Linux, w trakcie jego pracy, można usuwać różnorodną funkcjonalność (na przykład sterowniki urządzenia)},
		isTrue15=Tak, %
		answer15={Do kontroli pracy systemu można wykorzystać polecenia sysctl oraz zawartość katalogu /proc}, %
		isTrue16=Tak, %
		answer16={Do sprawdzenia jakie moduły załadowane są do jądra można wykorzystać polecenie lsmod}, %
		isTrue17=Tak, %
		answer17={Do załadowania modułu w jądrze można wykorzystać polecenia modprobe oraz insmod}, %
		isTrue18=Nie, %
		answer18={Do kontroli pracy systemu można wykorzystać polecenia sysctl oraz zawartość katalogu /var}
	}
	
	
	\item \question{Zaznacz wszystkie funkcje realizowane przez jądro monolityczne (na przykład jądro Linuxa)}
	{Tak}{Szeregowanie procesów}
	{Tak}{Zarządzanie pamięcią (zwalnianie/przydzielanie)}
	{Tak}{Szeregowanie I/O}
	{Tak}{Obsługa systemu plików}
	
	\newpage
	\item \question{Jakie operacje można wykonać za pomocą polecenia sysctl?}
	{Tak}{Ustawić wartości dla parametrów jądra}
	{Nie}{Ustawić wartości dla parametrów systemu plików}
	{Tak}{Wypisać wszystkie parametry jądra w trakcie działania systemu}
	{Nie}{Wypisać wszystkie parametry systemu plików}
	

	
	\item \questionVIII{%
		question={Polecenie sysctl:} %
	}{%
		isTrue1=Nie, %
		answer1={Służy do zmiany hasła użytkownika systemu}, %
		isTrue2=Nie, %
		answer2={Umożliwia zmianę nazwy użytkownika}, %
		isTrue3=Nie, %
		answer3={Wyświetla listę użytkowników w systemie}, %
		isTrue4=Tak, %
		answer4={Pozwala na zmianę parametrów jądra systemu w trakcie działania systemu operacyjnego}, %
		isTrue5=Tak, %
		answer5={To komenda pozwalająca na konfiguracje parametrów jądra systemu Linux.}, %
		isTrue6=Tak, %
		answer6={Wykonuje konfigurację jaką można także wykonać w wirtualnym systemie plików /proc/sys.}, %
		isTrue7=Nie, %
		answer7={Pozwala na rekompilację jądra z uwzględnieniem nowych plików konfiguracyjnych.}, %
		isTrue8=Nie, %
		answer8={Wyświetla wszystkie procesy w systemie.}.
	}
	
	\item \question{Wskaż prawdziwe zdania:}
	{Nie}{przy overcommit\_memory ustawionym na 2 system zawsze przydzieli aplikacjom dokładnie 100\% pamięci RAM}
	{Tak}{przy overcommit\_memory ustawionym na 1 możliwe jest uzyskanie za pomocą malloc() ilości pamięci wirtualnej większej niż objętość pamięci fizycznej + swap}
	{Tak}{przy overcommit\_memory ustawionym na 2 ilość pamięci przydzielonej aplikacjom zależy od overcommit\_ratio}
	{Nie}{kernel nigdy nie przydziela więcej pamięci niż jest dostępne fizycznie}
	
	\item \question{Sterowanie jądrem systemu Linux. Zaznacz poprawne odpowiedzi:}
	{Nie}{Nawet najdrobniejsza zmiana w pracy jądra systemu wymaga jego ponownej kompilacji.}
	{Tak}{Możliwa jest zmiana niektórych parametrów jądra w "locie" korzystając z komendy sysctl.}
	{Nie}{Po każdej zmianie parametru w jądrze systemu Linux należy ponownie uruchomić komputer.}
	{Tak}{Wartości sysctl wczytywane są podczas startu systemu z pliku /etc/sysct.conf.}
	
	\item \question{Sterowniki w systemach Linuxowych: Wskaż poprawne odpowiedzi.}
	{Tak}{Można wkompilować w jądro, ale można ładować dynamicznie bez potrzeby wkompilowywania.}
	{Tak}{Mogą być ładowane dynamicznie w trakcie działania systemu.}
	{Nie}{Są tylko wkompilowane w jądro i uruchamiane automatycznie. Nie ma innej możliwości instalacji i uruchomienia.}
	{Nie}{Po instalacji nowego sterownika zawsze wymagane jest ponowne uruchomienie komputera.}
	
	\item \question{W jaki sposóbm ożna wyłączyć partycję SWAP?}
	{Nie}{Nie można wyłączyć partycji SWAP}
	{Nie}{sudo setswap off}
	{Tak}{sudo swapoff -a}
	{Nie}{sudo swap stop}
	
	\item \question{Jakie jest zadanie jądra w systemie Linux?}
	{Tak}{Ładuje i odładowuje sterowniki urządzeń.}
	{Nie}{Tylko i wyłącznie zarządza pamięcią.}
	{Tak}{Pośredniczy pomiędzy aplikacją użytkownika a sprzętem.}
	{Tak}{Zarządza pamięcią.}
	
	\item \question{Jądro w systemie Linux odpowiedzialne jest za:}
	{Tak}{Sterowniki urządzeń}
	{Nie}{Wygląd interfejsu graficznego}
	{Tak}{Zarządzanie procesami}
	{Tak}{Obsługę pamięci}
	
	\item \question{Moduły jądra systemu Linux: wskaż wszystkie poprawne odpowiedzi.}
	{Tak}{Można pisać w języku C}
	{Nie}{Mogą być załadowane przez każdego użytkownika}
	{Tak}{Nie posiadają możliwości wyprowadzania danych na standardowe wyjście stdout za pomocą printf}
	{Tak}{Można je kompilować na tym samym systemie na którym zamierzamy je uruchomić.}
	
	\item \question{Co znajduje się w katalogu /proc/?}
	{Tak}{Informacje o procesach w systemie}
	{Nie}{Informacje o użytkownikach}
	{Tak}{Informacje o sieci}
	{Tak}{Ogólne informacje o systemie}
	
	\item \question{Program modprobe:}
	{Nie}{wymaga restartu aby zmiany zostały wprowadzone}
	{Tak}{umożliwia usuwanie modułów z kernela}
	{Tak}{umożliwia ładowanie modułów kernela}
	{Tak}{automatycznie dodaje moduły zależne}
	
	\item \question{Parametry jądra systemu Linux można odczytać za pomocą:}
	{Nie}{pliku /proc/stat}
	{Tak}{katalogu /proc/sys}
	{Nie}{komendy ps}
	{Tak}{komendy sysctl}
	
	\item \question{Które z poniższych komend sprawdza logi jądra systemu Linux}
	{Tak}{dmesg}
	{Nie}{klog}
	{Nie}{kmllg}
	{Nie}{kernelog}
	
	\newpage
	\item \question{Jądro systemu operacyjnego Linux:}%
	{Tak}{pośredniczy pomiędzy aplikacjami użytkownika, a sprzętem}%
	{Tak}{pośredniczy pomiędzy aplikacjami użytkownika, a pamięcią}%
	{Nie}{służy wyłącznie do uruchomienia systemu i skonfigurowania urządzeń, potem kończy swoją pracę}%
	{Nie}{NIE pozwala na ładowanie dodatkowych modułów}
	
	\item \question{Które ze zdań dotyczących sysctl jest poprawne?}%
	{Tak}{Katalog /proc/sys dostarcza interfejs do parametrów sysctl}%
	{Tak}{/proc/sys/vm/overcommit\_memory jest odpowiednikiem parametru vm.overcommit\_memory w sysctl.conf}%
	{Nie}{jeżeli katalog /proc/sys jest tylko do odczytu to da się mimo to zmieniać parametry przez komendę sysctl}%
	{Nie}{Wartości sysctl są wczytywane przy starcie systemu z /proc/sys/vm/sysctl.conf}
	
	\item \question{Zaznacz prawdziwe zdania dotyczące partycji wymiany (SWAP) w systemie Linux:}%
	{Tak}{Domyślnie jest na niej zapisywany zrzut pamięci RAM przy hibernacji}%
	{Tak}{Można go aktywować i dezaktywować podczas działania systemu}%
	{Nie}{Jest zamontowana w katalogu /swap}%
	{Nie}{Jest konieczna do działania systemu Linux}
	
	
	
\end{itemize}
	


\end{document}